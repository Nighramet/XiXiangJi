\documentclass{book}
\usepackage[fontset=none]{ctex}
\usepackage[a4paper, left=25mm, right=25mm, top=25mm, bottom=25mm]{geometry}

\xeCJKsetup{AutoFallBack=true}
% \setCJKmainfont{SimSun}[FallBack=SimSun-ExtB, BoldFont={Noto Sans CJK SC}, ItalicFont={Noto Sans CJK SC}]
\setCJKmainfont{NanoOldSong-A}[FallBack=NanoOldSong-B, AutoFakeBold, ItalicFont=NanoOldSong-A]

\newcommand\npart[1]{\part*{#1}\addcontentsline{toc}{part}{#1}}
\newcommand\nchapter[1]{\chapter*{#1}\markboth{#1}{}\addcontentsline{toc}{chapter}{#1}}

\title{西厢记}
\author{王实甫}
\date{元}

\begin{document}

\maketitle

\pagenumbering{Roman}
\tableofcontents

\newpage
\pagenumbering{arabic}
\npart{第一本\ 张君瑞闹道场杂剧}

\nchapter{楔子}

[外扮老夫人上开]老身姓郑,夫主姓崔,官拜前朝相国,不幸因病告殂。只生得个小姐,小字莺莺,年一十九岁,针黹女工,诗词书算,无不能者。老相公在日,曾许下老身之侄,乃郑尚书之长子郑恒为妻。因俺孩儿父丧未满,未得成合。又有个小妮子,是自幼伏侍孩儿的,唤做红娘。一个小厮儿,唤做欢郎。先夫弃世之后,老身与女孩儿扶柩至博陵安葬,因路途有阻,不能得去。来到河中府,将这灵柩寄在普救寺内。这寺是先夫相国修造的,是则天娘娘香火院,况兼法本长老,又是俺相公剃度的和尚,因此俺就这西厢下一座宅子安下。一壁写书附京师去,唤郑恒来,相扶回博陵去。我想先夫在日,食前方丈,从者数百,今日至亲则这三四口儿,好生伤感人也呵。

[仙吕][赏花时]夫主京师禄命终,子母孤孀途路穷,因此上旅榇在梵王宫。盼不到博陵旧冢,血泪洒杜鹃红。今日暮春天气,好生困人。不免唤红娘出来分付他。红娘何在?

[旦俫扮红见科]

[夫人云]你看佛殿上没人烧香呵,和小姐闲散心耍一回去来。

[红云]谨依严命。

[夫人下]

[红云]小姐有请。

[正旦扮莺莺上]

[红云]夫人着俺和姐姐佛殿上闲耍一回去来。

[旦唱][幺篇]可正是人值残春蒲郡东,门掩重关萧寺中。花落水流红,闲愁万种,无语怨东风。

[并下]

\nchapter{第一折}

[正末扮骑马引俫人上开]小生姓张名珙,字君瑞,本贯西洛人也。先人拜礼部尚书,不幸五旬之上因病身亡。后一年丧母。小生书剑飘零,功名未遂,游于四方。即今贞元十七年二月上旬,唐德宗即位,欲往上朝取应,路经河中府,过蒲关上,有一人姓杜名确,字君实,与小生同郡同学,当初为八拜之交,后弃文就武,遂得武举状元,官拜征西大元帅,统领十万大军,镇守着蒲关。小生就望哥哥一遭,却往京师求进。暗想小生萤窗雪案,刮垢磨光,学成满腹文章,尚在湖海飘零,何日得遂大志也呵!万金宝剑藏秋水,满马春愁压绣鞍。

[仙吕][点绛唇]游艺中原,脚跟无线,如蓬转。望眼连天,日近长安远。

[混江龙]向诗书经传,蠹鱼似不出费钻研。将棘围守暖,把铁砚磨穿。投至得云路鹏程九万里,先受了雪窗萤火二十年。才高难入俗人机,时乖不遂男儿愿。空雕虫篆刻,缀断简残编。行路之间,早到蒲津。这黄河有九曲,此正古河内之地,你看好形势也呵!

[油葫芦]九曲风涛何处显,则除是此地偏。这河带齐梁分秦晋隘幽燕。雪浪拍长空,天际秋云卷;竹索缆浮桥,水上苍龙偃;东西溃九州,南北串百川。归舟紧不紧如何见?却便似驽箭乍离弦。

[天下乐]只疑是银河落九天。渊泉云外悬,入东洋不离此径穿。滋洛阳千种花,润梁园万顷田,也曾泛浮槎到日月边。

话说间早到城中。这里一座店儿,琴童,接下马者。店小二哥那里?

[小二上云]自家是这状元店里小二哥。官人要下呵,俺这里有干净店房。

[末云]头房里下,先撒和那马者。小二哥你来,我问你:这里有甚么闲散心处?名山胜境、福地宝坊皆可。

[小二云]俺这里有一座寺,名曰普救寺,是则天皇后香火院,盖造非俗:琉璃殿相近青霄,舍利塔直侵云汉。南来北往,三教九流,过者无不瞻仰,则除那里可以君子游玩。

[末云]琴童,料持下响午饭,那里走一遭,便回来也。

[童云]安排下饭,撒和了马,等哥哥回家。[下]

[法聪上]小僧法聪,是这普救寺法本长老座下弟子。今日师父赴斋去了,着我在寺中,但有探长老的,便记着,待师父回来报知。山门下立地,看有甚么人来。

[末上云]却早来到也。

[见聪了,聪问云]客官从何来?

[末云]小生西洛至此,闻上刹幽雅清爽,一来瞻仰佛像,二来拜谒长老。敢问长老在么?

[聪云]俺师父不在寺中,贫僧弟子法聪的便是。请先生方丈拜茶。

[末云]既然长老不在呵,不必吃茶。敢烦和尚相引瞻仰一遭,幸甚。

[聪云]小僧取钥匙,开了佛殿、钟楼、塔院、罗汉堂、香积厨,盘桓一会,师父敢待回来。

[末云]是盖造得好也呵!

[村里迓鼓]随喜了上方佛殿,早来到下方僧院。行过厨房近西、法堂北、钟楼前面。游了洞房,登了宝塔,将回廊绕遍。数了罗汉,参了菩萨,拜了圣贤。

[莺莺引红娘捻花枝上云]红娘,俺去佛殿上耍去来。

[末做见科]呀!正撞着五百年前风流业冤。

[元和令]颠不刺的见了万千,似这般可喜娘的庞儿罕曾见。则着人眼花撩乱口难言,魂灵儿飞在半天。他那里尽人调戏亸着香肩,只将花笑捻。

[上马娇]这的是兜率宫,休猜做了离恨天。呀,谁想着寺里遇神仙!我见他宜嗔宜喜春风面,偏、宜贴翠花钿。

[胜葫芦]则见他宫样眉儿新月偃,斜侵入鬓云边。

[旦云]红娘,你觑:寂寂僧房人不到,满阶苔衬落花红。

[末云]我死也!未语人前先腼腆,樱桃红绽,玉粳白露,半晌恰方言。

[幺篇]恰便似呖呖莺声花外啭,行一步可人怜。解舞腰肢娇又软,千般袅娜,万般旖旎,似垂柳晚风前。

[红云]那壁有人,咱家去来。

[旦回顾觑末下]

[末云]和尚,恰怎么观音现来?

[聪云]休胡说!这是河中开府崔相国的小姐。

[末云]世间有这等女子,岂非天姿国色乎?休说那模样儿,则那一对小脚儿,价值百镒之金。

[聪云]偌远地,他在那壁,你在这壁,系着长裙儿,你便怎知他脚儿小?

[末云]法聪,来来来,你问我怎便知,你觑:

[后庭花]若不是衬残红芳径软,怎显得步香尘底样儿浅。且休题眼角儿留情处,则这脚踪儿将心事传。慢俄延,投至到栊门儿前面,刚那了一步远。刚刚的打个照面,风魔了张解元。似神仙归洞天,空馀下杨柳烟,只阙得鸟雀喧。

[柳叶儿]呀,门掩着梨花深院,粉墙儿高似青天。恨天、天不与人行方便,好着我难消遣,端的是怎留连。小姐呵,则被你兀的不引了人意马心猿。

[聪云]休惹事,河中开府的小姐去远了也。

[末唱][寄生草]兰麝香仍在,佩环声渐远。东风摇曳垂杨线,游丝牵惹桃花片,珠帘掩映芙蓉面。你道是河中开府相公家,我道是南海水月观音现。

``十年不识君王面,恰信婵娟解误人。''小生便不往京师去应举也罢。

[觑聪云]敢烦和尚对长老说知,有僧房借半间,早晚温习经史,胜如旅邸内冗杂。房金依例拜纳。小生明日自来也。

[赚煞]饿眼望将穿,馋口涎空咽,空着我透骨髓相思病染,怎当他临去秋波那一转。休道是小生,便是铁石人也意惹情牵。近庭轩花柳争妍,日午当庭塔影圆。春光在眼前,争奈玉人不见,将一座梵王宫疑是武陵源。[下]

\nchapter{第二折}

[夫人上白]前日长老将钱去与老相公做好事,不见来回话。道与红娘,传着我的言语,去问长老,几时好与老相公做好事?就着他办下东西的当了,来回我话者。[下]

[净扮洁上]老僧法本,在这普救寺内做长老。此寺是则天皇后盖造的,后来崩损,又是崔相国重修的。现今崔老夫人领着家眷,扶柩回博陵,因路阻暂寓本寺西厢之下,待路通回博陵迁葬。老夫人处事温俭,治家有方,是是非非,人莫敢犯。夜来老僧赴斋,不知曾有人来望老僧否?[唤聪问科]

[聪云]夜来有一秀才,自西洛而来,特谒我师,不遇而返。

[洁云]山门外觑着,若再来时,报我知道。

[末上云]昨日见了那小姐,到有顾盼小生之意。今日去问长老借一间僧房,早晚温习经史;倘遇那小姐出来,必当饱看一会。

[中吕][粉蝶儿]不做周方,埋怨杀你个法聪和尚。借与我半间儿客舍僧房,与我那可憎才居止处门儿相向。虽不能勾窃玉偷香,且将这盼行云眼睛儿打当。

[醉春风]往常时见傅粉的委实羞,画眉的敢是谎。今日多情人一见了有情娘,着小生心儿里早痒、痒。迤逗得肠荒,断送得眼乱,引惹得心忙。

[末见聪科]

[聪云]师父正望先生来哩,只此少待,小僧通报去。

[洁出见末科]

[末云]是好一个和尚呵!

[迎仙客]我则见他头似雪,鬓如霜,面如童,少年得内养。貌堂堂,声朗朗,头直上只少个圆光,却便似捏塑来的僧伽像。

[洁云]请先生方丈内相见。夜来老僧不在,有失迎迓,望先生恕罪。

[末云]小生久闻老和尚清誉,欲来座下听讲,何期昨日不得相遇。今能一见,是小生三生有幸矣。

[洁云]先生世家何郡?敢问上姓大名,因甚至此?

[末云]小生姓张名珙,字君瑞。

[石榴花]大师一一问行藏,小生仔细诉衷肠。自来西洛是吾乡,宦游在四方,寄居咸阳。先人拜礼部尚书多名望,五旬上因病身亡。

[洁云]老相公弃世,必有所遗。

[末唱]平生正直无偏向,止留下四海一空囊。

[洁云]老相公在官时浑俗和光。

[末唱][斗鹌鹑]俺先人甚的是浑俗和光,衠一味风清月朗。

[洁云]先生此一行,必上朝取应去。

[末唱]小生无意求官,有心待听进。小生特谒长老,奈路途奔驰,无以相馈——量着穷秀才人情则是纸半张。又没甚七青八黄,尽着你说短论长,一任待掂斤播两。径禀:有白银一两,与常住公用,略表寸心,望笑留是幸。

[洁云]先生客中,何故如此?

[末云]物鲜不足辞,但充讲下一茶耳。

[上小楼]小生特来见访,大师何须谦让。

[洁云]老僧决不敢受。

[末唱]这钱也难买柴薪,不够斋粮,且备茶汤。

[觑聪云]这一两银,为厚礼。你若有主张,对艳妆,将言词说上,我将你众和尚死生难忘。

[洁云]先生必有所请。

[末云]小生不揣有恳。因恶旅邸冗杂,早晚难以温习经史,欲假一室,晨昏听讲,房金按月任意多少。

[洁云]敝寺颇有数间,任先生拣选。

[末唱][幺篇]也不要香积厨,枯木堂。远着南轩,离着东墙,靠着西厢。近主廊,过耳房,都皆停当。

[洁云]便不呵,就与老僧同处何如?

[末笑云]要恁怎么?你是必休提着长老方丈。

[红上云]老夫人着俺问长老,几时好与老相公做好事,看得停当回话。须索走一遭去来。

[见洁科]长老万福。夫人使侍妾来问,几时好与老相公做好事,着看的停当了回话。

[末背云]好个女子也呵!

[脱布衫]大人家举止端详,全没那半点儿轻狂。大师行深深拜了,启朱唇语言得当。

[小梁州]可喜娘的庞儿浅淡妆,穿一套缟素衣裳。胡伶渌老不寻常,偷睛望,眼挫里抹张郎。

[幺篇]若共他多情小姐同鸳帐,怎舍得他叠被铺床。我将小姐央,夫人怏,他不令许放,我亲自写与从良。

[洁云]二月十五日可与老相公做好事。

[红云]妾与长老同去佛殿看了,却回夫人话。

[洁云]先生请少坐,老僧同小娘子看一遭便来。

[末云]何故却小生?便同行一遭,又且何如?

[洁云]便同行。

[末云]着小娘子先行,俺近后些。

[洁云]一个有道理的秀才。

[末云]小生有一句话说,敢道么?

[洁云]便道不妨。

[末唱][快活三]崔家女艳妆,莫不是演撒你个老洁郎?

[洁云]俺出家人那有此事?

[末]既不沙,却怎睃趁着你头上放毫光?打扮的特来晃。

[洁云]先生是何言语!早是那小娘子不听得哩,若知呵,是甚意思!

[红上佛殿科]

[末唱][朝天子]过得主廊,引入洞房,好事从天降。我与你看着门儿,你进去。

[洁怒云]先生,此非先王之法言!岂不得罪于圣人之门乎?老僧偌大年纪,焉肯作此等之态?

[末唱]好模好样忒莽撞,没则罗便罢,烦恼则么耶唐三藏?怪不得小生疑你,偌大一个宅堂,可怎生别没个儿郎,使得梅香来说勾当。

[洁云]老夫人治家严肃,内外并无一个男子出入。

[末背云]这秃厮巧说!你在我行、口强,硬抵着头皮撞。

[洁对红云]这斋供道场都完备了,十五日请夫人小姐拈香。

[末问云]何故?

[洁云]这是崔相国小姐至孝,为报父母之恩,又是老相公禫日,就脱孝服,所以做好事。

[末哭科云]``哀哀父母,生我劬劳,欲报深恩,昊天罔极。''小姐是一女子,尚然有报父母之心;小生湖海飘零数年,自父母下世之后,并不曾有一陌纸钱相报。望和尚慈悲为本,小生亦备钱五千,怎生带得一分儿斋,追荐俺父母咱。便夫人知,也不妨,以尽人子之心。

[洁云]法聪,与这先生带一分者。

[末背问聪云]那小姐明日来么?

[聪云]他父母的勾当,如何不来?

[末背云]这五千钱使得有些下落者!

[四边静]人间天上,看莺莺强如做道场。软玉温香,休道是相亲傍,若能勾汤他一汤,到与人消灾障。

[洁云]都到方丈吃茶。

[做到科]

[末云]小生更衣咱。

[末出科云]那小娘子已定出来也,我则在这里等待问他咱。

[红辞洁云]我不吃茶了,恐夫人怪来迟,去回话也。

[红出科]

[末迎红娘祗揖科]小娘子拜揖。

[红云]先生万福。

[末云]小娘子莫非莺莺小姐的侍妾么?

[红云]我便是。何劳先生动问?

[末云]小生姓张,名珙,字君瑞,本贯西洛人也。年方二十三岁,正月十七日子时建生。并不曾娶妻……

[红云]谁问你来?

[末云]敢问小姐常出来么?

[红怒云]先生是读书君子,孟子曰:``男女授受不亲,礼也。''君知``瓜田不纳履,李下不整冠''。道不得个``非礼勿视,非礼勿听,非礼勿言,非礼勿动。''俺夫人治家严肃,有冰霜之操。内无应门五尺之童,年至十二三者,非呼召,不敢辄入中堂。向日莺莺潜出闺房,夫人窥之,召立莺莺于庭下,责之曰:``汝为女子,不告而出闺门,倘遇游客小僧私视,岂不自耻?''莺立谢而言曰:``今当改过从新,毋敢再犯。''是他亲女,尚然如此,可况以下侍妾乎!先生习先王之道,尊周公之礼,不干己事,何故用心?早是妾身,可以容恕。若夫人知其事呵,决无干休!今后得问的问,不得问的休胡说![下]

[末云]这相思索是害也。

[哨遍]听说罢心怀悒怏,把一天愁都撮在眉尖上。说``夫人节操凛冰霜,不召乎,谁敢辄入中堂!''自思想,比及你心儿里畏惧老母亲威严,小姐呵,你不合临去也回头望。待扬下教人怎扬?赤紧的情沾了肺腑,意惹了肝肠。若今生难得有情人,是前世烧了断头香。我得时节手掌儿里奇擎,心坎儿里温存,眼皮儿上供养。

[耍孩儿]当初那巫山远隔如天样,听说罢又在巫山那厢。业身躯虽是立在回廊,魂灵儿已在他行。本待要安排心事传幽客,我只怕漏泄春光与乃堂。夫人怕女孩儿春心荡,怪黄莺儿作对,怨粉蝶儿成双。

[五煞]小姐年纪小,性气刚。张郎倘得相亲傍,乍相逢厌见何郎粉,看邂逅偷将韩寿香。才到是未得风流况,成就了会温存的娇婿,怕甚么能拘束的亲娘。

[四煞]夫人忒虑过,小生空妄想。郎才女貌合相仿。休直待眉儿浅淡思张敞,春色飘零忆阮郎。非是咱自夸奖,他有德言工貌,小生有恭俭温良。

[三煞]想着他眉儿浅浅描,脸儿淡淡妆,粉香腻玉搓咽项。翠裙鸳绣金莲小,红袖鸾销玉笋长。不想呵其实强,你撇下半天风韵,我拾得万种思量。却忘了辞长老。

[见洁科]小生敢问长老:房舍如何?

[洁云]塔院侧边西厢一间房,甚是潇洒,正可先生安下,见收拾下了,随先生早晚来。

[末云]小生便回店中搬去。

[洁云]既然如此,老僧准备下斋,先生是必便来。[下]

[末云]若在店中人闹,到好消遣;搬在寺中静处,怎么捱这凄凉也呵!

[二煞]院宇深,枕簟凉。一灯孤影摇书幌。纵然酬得今生志,着甚支吾此夜长!睡不着如翻掌,少可有一万声长吁短叹,五千遍倒枕捶床。

[尾]娇羞花解语,温柔玉有香。我和他乍相逢记不真娇模样,我则索手抵着牙儿慢慢的想。[下]

\nchapter{第三折}

[正旦上云]老夫人着红娘问长老去了,这小贱人不来我行回话。

[红上云]回夫人话了,去回小姐话去。

[旦云]使你问长老,几时做好事?

[红云]恰回夫人话也,正待回姐姐话。二月十五日请夫人、姐姐拈香。

[红笑云]姐姐,你不知,我对你说一件好笑的的勾当。咱前日寺里见的那秀才,今日也在方丈里。他先出门儿外,等着红娘,深深唱个喏道:``小生姓张,名珙,字君瑞,本贯西洛人也,年二十三岁,正月十七日子时建生,并不曾娶妻。''姐姐,却是谁问他来?他又问:``那壁小娘子,莫非莺莺小姐的侍妾乎?小姐常出来么?''被红娘抢白了一顿呵回来了。姐姐,我不知他想甚么哩,世上有这等傻角!

[旦笑云]红娘,休对夫人说。天色晚也,安排香案,咱花园内烧香去来。[下]

[末上云]搬至寺中,正近西厢居址。我问和尚每来,小姐每夜花园内烧香。这个花园,和俺寺中合着。比及小姐出来,我先在太湖石畔墙角儿边等待,饱看一会。两廊僧众都睡着了,夜深人静,月朗风清,是好天气也呵!正是:闲寻方丈高僧语,闷对西厢皓月吟。

[越调][斗鹌鹑]玉宇无尘,银河泻影,月色横空,花阴满庭。罗袂生寒,芳心自警。侧着耳朵儿听,蹑着脚步儿行:悄悄冥冥,潜潜等等。

[紫花儿序]等待那齐齐整整,袅袅婷婷,姐姐莺莺。一更之后,万籁无声,直至莺庭。若是回廊下没揣的见俺可憎,将他来紧紧的搂定;则问你那会少离多,有影无形。

[旦引红娘上云]开了角门儿,将香桌出来者。

[末唱][金蕉叶]猛听得角门儿呀的一声,风过处花香细生。踮着脚尖儿仔细定睛,比我那初见时庞儿越整。

[旦云]红娘,移香桌儿,近太湖石畔放者。

[末做看科云]料想春娇厌拘束,等闲飞出广寒宫。看他容分一捻,体露半襟,亸香袖以无言,垂罗裙而不语。似湘陵妃子,斜倚舜庙朱扉;如玉殿嫦娥,微现蟾宫素影。是好女子也呵!

[调笑令]我这里甫能、见娉婷,比着那月殿嫦娥也不恁般撑。遮遮掩掩穿芳径,料应来小脚儿难行。可喜娘的脸儿百媚生,兀的不引了人魂灵!

[旦云]取香来。

[末云]听小姐祝告甚么。

[旦云]此一柱香,愿化去先人,早生天界;此一柱香,愿堂中老母,身安无事;此一柱香……[做不语科]

[红云]姐姐不祝这一柱香,我替姐姐祝告:愿俺姐姐早寻一个姐夫,拖带红娘咱!

[旦再拜云]心中无限伤心事,尽在深深两拜中。[长吁科]

[末云]小姐倚栏长叹,似有动情之意。

[小桃红]夜深香霭散空庭,帘幕东风静。拜罢也斜将曲栏凭,长吁了两三声。剔团圞明月如悬镜,又不是轻云薄雾,都则是香烟人气,两般儿氤氲得不分明。

我虽不及司马相如,我则看小姐颇有文君之意。我且高吟一绝,看他则甚:

月色溶溶夜,花阴寂寂春。如何临皓魄,不见月中人?

[旦云]有人墙角吟诗。

[红云]这声音,便是那二十三岁不曾娶妻的那傻角。

[旦云]好清新之诗!我依韵做一首。

[红云]你两个是好做一首!

[旦念诗云]兰闺久寂寞,无事度芳春。料得行吟者,应怜长叹人。

[末云]好应酬得快也呵!

[秃厮儿]早是那脸儿上扑堆着可憎,那堪那心儿里埋没着聪明。他把那新诗和得忒应声,一字字诉衷情,堪听。

[圣药王]那语句清,音律轻,小名儿不枉了唤做莺莺。他若是共小生、厮觑定,隔墙儿酬和到天明,方信道惺惺的自古惜惺惺。我撞出去,看他说甚么。

[麻郎儿]我拽起罗衫欲行,[旦做见科]他陪着笑脸儿相迎。不做美的红娘忒浅情,便做道谨依来命。

[红云]姐姐,有人!咱家去来,怕夫人嗔着。

[莺回顾下]

[末唱][幺篇]我忽听、一声、猛惊,原来是扑刺刺宿鸟飞腾,颤巍巍花梢弄影,乱纷纷落红满径。小姐你去了呵,那里发付小生!

[络丝娘]空撇下碧澄澄苍苔露冷,明皎皎花筛月影。白日凄凉枉耽病,今夜把相思再整。

[东原乐]帘垂下,户已扃。却才个悄悄相问,他那里低低应。月朗风清恰二更,厮工徯幸,他无缘,小生薄命。

[绵搭絮]恰寻归路,伫立空庭,竹梢风摆,斗柄云横。呀,今夜凄凉有四星,他不偢人待怎生!虽然是眼角传情,咱两个口不言心自省。今夜甚睡到得我眼里呵!

[拙鲁速]对着盏碧荧荧短檠灯,倚着扇泠清清旧帏屏。灯儿又不明,梦儿又不成;窗儿外淅零零的风儿透疏棂,忒楞楞的纸条儿鸣;枕头儿上孤另,被窝儿里寂静。你便是铁石人,铁石人也动情。

[幺篇]怨不能,恨不成,坐不安,睡不宁。有一日柳遮花映,雾障云屏,夜阑人静,海誓山盟——恁时节风流嘉庆,锦片也似前程;美满恩情,咱两个画堂春自生。

[尾]一天好事从今定,一首诗分明照证。再不向表琐闼梦儿中寻,则去那碧桃花树儿下等。[下]

\nchapter{第四折}

[洁引聪上云]今日二月十五日开启,众僧动法器者!请夫人小姐拈香。比及夫人未来,先请张生拈香,怕夫人问呵,则说道贫僧亲者。

[末上云]今日二月十五日,和尚请拈香,须索走一遭。

[双调][新水令]梵王宫殿月轮高,碧琉璃瑞烟笼罩。香烟云盖结,讽咒海波潮。幡影飘飖,诸檀越尽来到。

[驻马听]法鼓金铎,二月春雷响殿角;钟声佛号,半天风雨洒松梢。侯门不许老僧敲,纱窗外定有红娘报。害相思的馋眼脑,见他时须看个十分饱。

[末见洁科]

[洁云]先生先拈香,恐夫人问呵,则说是老僧的亲。

[末拈香科][沈醉东风]惟愿存在的人间寿高,亡化的天上逍遣。为曾祖父先灵,礼佛法僧三宝。焚名香暗中祷告:则愿得红娘休劣,夫人休焦,犬儿休恶。佛啰,早成就了幽期密约。

[夫人引旦上云]长老请拈香,小姐,咱走一遭,

[末做见科]

[觑聪云]为你志诚呵,神仙下降也。

[聪云]这生却早两遭儿也。

[末唱][雁儿落]我则道这玉天仙离了碧霄,原来是可意种来清醮。小子多愁多病身,怎当他倾国倾城貌。

[得胜令]恰便似檀口点樱桃,粉鼻儿倚琼瑶。淡白梨花面,轻盈杨柳腰。妖娆,满面儿扑堆着俏;苗条,一团儿衠是娇。

[洁云]贫僧一句话,夫人行敢道么?老僧有个敝亲,是个饱学的秀才,父母亡后,无可相报。对我说,央及带一分斋,追荐父母。贫僧一时应允了,恐夫人见责。

[夫人云]长老的亲,便是我的亲,请来厮见咱。

[末拜夫人科]

[众僧见旦发科]

[乔牌儿]大师年纪老,法座上也凝眺;举名的班首真呆僗,觑着法聪头做金磬敲。

[甜水令]老的小的,村的俏的,没颠没倒,胜似闹元宵。稔色人儿,可意冤家,怕人知道,看时节泪眼偷瞧。

[折桂令]着小生迷留没乱,心痒难挠。哭声儿似莺啭乔林,泪珠儿似露滴花梢。大师也难学,把一个发慈悲的脸儿来朦着。击磬的头陀懊恼,添香的行者心焦。烛影风摇,香霭云飘,贪看莺莺,烛灭香消。

[洁云]风灭灯也。

[末云]小生点灯烧香。

[旦与红云]那生忙了一夜。

[锦上花]外像儿风流,青春年少;内性儿聪明,冠世才学。扭捏着身子儿百般做作,来往向人前卖弄俊俏。

[红云]我猜那生——黄昏这一回,白日那一觉,窗儿外那会镬铎,到晚来向书帏里比及睡着,千万声长吁捱不到晓。

[末云]那小姐好生顾盼小子!

[碧玉箫]情引眉梢,心绪你知道;愁种心苗,情思我猜着。畅懊恼,响铛铛云板敲,行者又嚎,沙弥又哨,恁须不夺人之好。

[洁与众僧发科]

[动法器了,洁摇铃跪宣疏了,烧纸科]

[洁云]天明了也,请夫人小姐回宅。

[末云]再做一会也好,那里发付小生也呵!

[鸳鸯煞]有心争似无心好,多情却被无情恼。劳攘了一宵,月儿沉,钟儿响,鸡儿叫。唱道是玉人归去得疾,好事收拾得早。道场毕诸人散了,酩子里各归家,葫芦提闹到晓。

[并下]

[络丝娘煞尾]则为你闭月羞花相貌,少不得剪草除根大小。

\newpage

题目:老夫人闭春院\ 崔莺莺烧夜香

正名:小红娘传好事\ 张君瑞闹道场

\npart{第二本\ 崔莺莺夜听琴杂剧}

\nchapter{第一折}

[净扮孙飞虎上开]自家姓孙,名彪,字飞虎。方今上德宗即位,天下扰攘。因主将丁文雅失政,俺分统五千人马,镇守河桥。近知先相公崔珏之女莺莺,眉黛青颦,莲脸生春,有倾国倾城之容,西子太真之颜,见在河中府普救寺借居。我心中想来,当今用武之际,主将尚然不正,我独廉何为?大小三军,听吾号令:人尽衔枚,马皆勒口,连夜进兵河中府,掳莺莺为妻,是我平生愿足!

[法本慌上]谁想孙飞虎将半万贼兵,围住寺门,鸣锣击鼓,呐喊摇旗,欲掳莺莺小姐为妻。我今不敢违误,即索报知夫人走一遭。[下]

[夫人上慌云]如此却怎了?俺同到小姐卧房里商量去。[下]

[旦引红上云]自见了张生,神魂荡漾,情思不快,茶饭少进。早是离人伤感,况值暮春天道,好烦恼人也呵!好句有情联夜月,落花无语怨东风。

[仙吕][八声甘州]恹恹瘦损,早是伤神,那值残春。罗衣宽褪,能消几度黄昏?风袅篆烟不卷帘,雨打梨花深闭门;无语凭阑干,目断行云。

[混江龙]落红成阵,风飘万点正愁人;池塘梦晓,阑槛辞春。蝶粉轻沾飞絮雪,燕泥香惹落花尘。系春心情短柳丝长,隔花阴人远天涯近。香消了六朝金粉,清减了三楚精神。

[红云]姐姐情思不快,我将被儿薰得香香的,睡些儿。

[旦唱][油葫芦]翠被生寒压绣裀,休将兰麝薰;便将兰麝薰尽,则索自温存。昨宵个锦囊佳制明勾引,今日个玉堂人物难亲近。这些时坐又不安,睡又不稳,我欲待登临又不快,闲行又闷,每日价情思睡昏昏。

[天下乐]红娘呵,我则索搭伏定鲛绡枕头儿上盹,但出闺门,影儿般不离身。

[红云]不干红娘事,老夫人着我跟着姐姐来。

[旦云]俺娘也好没意思。这些时直恁般堤防着人!小梅香伏侍得勤,老夫人拘系得紧,则怕俺女孩儿折了气分。

[红云]姐姐往常不曾如此无情无绪,自曾见了那张生,便觉心事不宁,却是如何?

[旦唱][那吒令]往常但见个外人,氲的早嗔;但见个客人,厌的倒褪;从见了那人,兜的便亲。想着他昨夜诗,依前韵,酬和得清新。

[鹊踏枝]吟得句儿匀,念得字儿真,咏月新诗,煞强似织锦回文。谁肯把针儿将线引,向东邻通个殷勤。

[寄生草]想着文章士,旖旎人。他脸儿清秀身儿俊,性儿温克情儿顺,不由人口儿里作念心儿里印。学得来一天星斗焕文章,不枉了十年窗下无人问。

[飞虎领兵上围寺科][下]

[卒子内高叫云]寺里人听者:限你每三日内,将莺莺献出来,与俺将军成亲,万事干休。三日之后不送出,伽蓝尽皆焚烧,僧俗寸斩,不留一个。

[夫人洁同上,敲门了,红看了云]姐姐,夫人和长老都在房门前。

[旦见了科]

[夫人云]孩儿,你知道么,如今孙飞虎将半万贼兵,围住寺门,道你眉黛青颦,莲脸生春,似倾国倾城的太真,要掳你做压寨夫人。孩儿,怎生是了也?

[旦唱][六玄序]听说罢魂离了壳,现放着祸灭身。将袖梢儿揾不住啼痕。好教我去住无因,进退无门。可着俺那埚儿里人急偎亲?孤孀子母无投奔,赤紧的先亡过了有福之人。耳边厢金鼓连天振,征云冉冉,土雨纷纷。

[幺篇]那厮每风闻,胡云,道我眉黛青颦,莲脸生春,恰便似倾国倾城的太真。兀的不送了他三百僧人!半万贼军,半霎儿敢剪草除根。这厮每于家为国无忠信,恣情的掳掠人民。更将那天宫般盖造焚烧尽,则没那诸葛孔明,便待要博望烧屯。

[夫人云]老身年六十年,不为寿夭;奈孩儿年少,未得从夫,却如之奈何?

[旦云]孩儿有一计:想来则是将我与贼汉为妻,庶可免一家儿性命。

[夫人哭云]俺家无犯法之男,再婚之女,怎舍得你献与贼汉,却不辱没了俺家谱?

[洁云]俺同到法堂两廊下,问僧俗有高见者,俺一同商议个长便。

[同到法堂科]

[夫人云]小姐,却是怎生?

[旦云]不如将我与贼人,其便有五。

[后庭花]第一来免摧残老太君;第二来免堂殿作灰烬;第三来诸僧无事得安存;第四来先君灵柩稳;第五来欢郎虽是未成人,

[欢云]俺呵,打甚么不紧。

[旦唱]须是崔家后代孙。莺莺为惜己身,不行从着乱军,诸僧众污血痕,将伽蓝火内焚,先灵为细尘,断绝了爱弟亲,割开了慈母恩。

[柳叶儿]呀,将俺一家儿不留一个龆龀。待从军又怕辱没了家门,我不如白练套头儿寻个自尽,将我尸榇,献与贼人,也须得个远害全身。

[青歌儿]母亲,都做了莺莺生忿,对旁人一言难尽。母亲,休爱惜莺莺这一身。恁孩儿别有一计:不拣何人,建立功勋,杀退贼军,扫荡妖氛,倒陪家门,情愿与英雄结婚姻,成秦晋。

[夫人云]此计较可。虽然不是门当户对,也强如陷于贼中。长老,在法堂上高叫:两廊僧俗,但有退兵之策的,倒陪房奁,断送莺莺与他为妻。

[洁叫了,住]

[末鼓掌上云]我有退兵之策,何不问我?[见夫人了]

[洁云]这秀才便是前日带追荐的秀才。

[夫人云]计将安在?

[末云]重赏之下,必有勇夫;赏罚若明,其计必成。

[旦背云]只愿这生退了贼者。

[夫人云]恰才与长老说下,但有退得贼兵的,将小姐与他为妻。

[末云]即是恁的,休唬了我浑家,请入卧房里去,俺自有退兵之策。

[夫人云]小姐和红娘回去者。

[旦对红云]难得此生这一片好心。

[赚煞]诸僧众各逃生,从家眷谁偢问。这生不相识横枝儿着紧。非是书生多议论,也堤防着玉石俱焚。虽然是不关亲,可怜见命在逡巡。济不济权将秀才来尽。果若有出师表文,吓蛮书信,张生呵,则愿得笔尖儿横扫了五千人。

\nchapter{楔子}

[夫人云]此事如何?

[末云]小生有一计,先用着长老。

[洁云]老僧不会厮杀,请秀才别换一个。

[末云]休慌,不要你厮杀。你出去与贼汉说:``夫人本待便将小姐出来,送与将军,奈有父丧在身。不争鸣锣击鼓,惊死小姐,也可惜了。将军若要做女婿呵,可按甲束兵,退一射之地。限三日功德圆满,脱了孝服,换上颜色衣服,倒陪房奁,定将小姐送与将军。不争便送来,一来父服在身,二来于军不利。''你去说来。

[洁云]三日后如何?

[末云]有计在后。

[洁朝鬼门道叫科]请将军打话。

[飞虎卒上云]快送出莺莺来!

[洁云]将军息怒。夫人使老僧来与将军说。

[说如前了]

[飞虎云]既然如此,限你三日后若不送来,我着你人人皆死,个个不存。你对夫人说去:恁的这般好性儿的女婿,教他招了者!

[洁云]贼兵退了也,三日后不送出去,便都是死的。

[末云]小子有一故人,姓杜,名确,号为白马将军,现统十万大兵,镇守着蒲关。一封书去,此人必来救我。此间离蒲关四十五里,写了书呵,怎得人送去?

[洁云]若是白马将军肯来,何虑孙飞虎!俺这里有一个徒弟,唤作惠明,则是要吃酒厮打。若使央他去,定不肯去;须将言语激他,他便去。

[末唤云]有书寄与杜将军,谁敢去?谁敢去?

[惠明上云]我敢去!

[正宫][端正好]不念《法华经》,不礼《梁皇忏》,颩了僧伽帽,袒下我这偏衫,杀人心逗起英雄胆,两只手将乌龙尾钢椽𭡗。

[滚绣球]非是我贪,不是我敢,知他怎生唤做打参,不踏步直杀出虎窟龙潭。非是我搀,不是我揽,这些时吃菜馒头委实口淡,五千人也不索灸煿煎爁。腔子里热血权消渴,肺腑内生心且解馋,有甚腌臜!

[叨叨令]浮沙羹宽片粉添些杂糁;酸黄韭烂豆腐休调啖。万余斤黑面从教暗,我将这五千人做一顿馒头馅。是必休误了也么哥,休误了也么哥!包残余肉把青盐蘸。

[洁云]张秀才着你寄书去蒲关,你敢去么?

[惠唱][倘秀才]你那里问小僧敢去也那不敢,我这里启大师用咱也不用咱。你道是飞虎将声名播斗南;那厮能淫欲,会贪婪,诚何以堪!

[末云]你是出家人,却怎不看经礼忏,则厮打为何?

[惠唱][滚绣球]我经文也不会谈,逃禅也懒去参;戒刀头近新来钢蘸,铁棒上无半星儿土渍尘缄。别的都僧不僧、俗不俗,女不女、男不男,则会斋的饱也则向那僧房中胡渰,那里怕焚烧了兜率伽蓝。则为那善文能武人千里,凭着这济困扶危书一缄,有勇无渐。

[末云]他倘不放你过去,如何?

[惠云]他不放我呵,你放心。

[白鹤子]着几个小沙弥把幢幡宝盖擎,壮行者将捍棒镬叉担。你排阵脚将众僧安,我撞钉子把贼兵来探。

[二]远的破开步将铁棒颩,近的顺着手把戒刀钐;有小的提起来将脚尖𨀵,有大的扳下来把髑髅勘。

[一]瞅一瞅古都都翻了海波,滉一滉厮琅琅振动山岩;脚踏得赤力力地轴摇,手扳得忽剌剌天关撼。

[耍孩儿]我从来驳驳劣劣,世不曾忑忑忐忐,打熬成不厌天生敢。我从来斩钉截铁常居一,不似恁惹草拈花没掂三。劣性子人皆惨,舍着命提刀仗剑,更怕甚勒马停骖。

[二]我从来欺硬怕软,吃苦不甘,你休只因亲事胡扑掩。若是杜将军不把干戈退,张解元干将风月担,我将不志诚的言词赚。倘或纰缪,倒大羞惭。

[惠云]将书来,你等回音者。

[收尾]恁与我助威风擂几声鼓,仗佛力呐一声喊。绣旗下遥见英雄俺,我教那半万贼兵唬破胆。[下]

[末云]老夫人长老都放心,此书到日,必有佳音。咱眼观旌节旗,耳听好消息。你看一封书札逡巡至,半万雄兵咫尺来。

[并下]

[杜将军引卒子上开]林下晒衣嫌日淡,池中濯足恨鱼腥;花根本艳公卿子,虎体原班将相孙。自家姓杜,名确,字君实,本贯西洛人也。自幼与君瑞同学儒业,后弃文就武。当年武举及第,官拜征西大将军,正授管军元帅,统领十万之众,镇守着蒲关。有人自河中来,听知君瑞兄弟在普救寺中,不来望我;着人去请,亦不肯来,不知主甚意。今闻丁文雅失政,不守国法,剽掠黎民。我为不知虚实,未敢造次兴师。孙子曰:``凡用兵之法,将受命于君,合军聚众,圯地无舍,衢地交合,绝地无留;围地则谋,死地则战;途有所不由,军有所不击,城有所不攻,地有所不争,君命有所不受。故将通于九变之利者,知用兵矣。治兵不知九变之术,虽知五利,不能得人用矣。''吾之未疾进兵征讨者,为不知地利浅深出没之故也。昨日探听去,不见回报。今日升帐,看有甚军情,来报我知道者。

[卒子引惠明和尚上开]

[惠明云]我离了普救寺,一日至蒲关,见杜将军走一遭。

[卒报科]

[将军云]着他过来!

[惠打问讯了云]贫僧是普救寺僧。今有孙飞虎作乱,将半万贼兵,围往寺门,欲劫故臣崔相国女为妻。有游客张君瑞奉书,令小僧拜投于麾下,欲求将军以解倒悬之危。

[将军云]将书过来。

[惠投书了]

[将军拆书念曰]``珙顿首再拜大元帅将军契兄纛下:伏自洛中,拜违犀表,寒暄屡隔,积有岁月,仰德之私,铭刻如也。忆昔联床风雨,叹今彼各天涯;客况复生于肺腑,离愁无慰于羁怀。念贫处十年藜藿,走困他乡;羡威统百万貔貅,坐安边境。故知虎体食天禄,瞻天表,大德胜常;使贱子慕台颜,仰台翰,寸心为慰。辄禀:小弟辞家,欲诣帐下,以叙数载间阔之情;奈至河中府普救寺,忽值采薪之忧。不期有贼将孙飞虎,领兵半万,欲劫故臣崔相国之女,实为迫切狼狈。小弟之命,亦在逡巡。万一朝廷知道,其罪何归?将军倘不弃旧交之情,兴一旅之师,上以报天子之恩,下以救苍生之急;使故相国虽在九泉,亦不泯将军之德。愿将军虎视去书,使小弟鹄观来旄。造次干渎,不胜惭愧。伏乞台照不宣。张珙再拜,二月十六日书''

[将军云]既然如此,和尚你行,我便来。

[惠明云]将军是必疾来者。

[将军云]虽无圣旨发兵,将在军,君命有所不受。大小三军,听吾将令:速点五千人马,人尽衔枚,马皆勒口。星夜起发,直至河中府普救寺,救张生走一遭。

[飞虎引卒子上开]

[将军引卒子骑竹马调阵拿绑下]

[夫人洁同末上云]下书已两日,不见回音。

[末云]山门外呐喊摇旗,莫不是俺哥哥至了?

[末见将军了][引夫人拜了]

[将军云]杜确有失防御,致令老夫人受惊,切忽见罪是幸。

[末拜将军了]自别兄长台颜,一向有失听教。今得一见,如拨云睹日。

[夫人云]老身子母,如将军所赐之命,将何补报?

[将军云]不敢,此乃职分之所当为。敢问贤弟:因甚不至戎帐?

[末云]小弟欲来,奈小疾偶作,不能动止,所以失敬。今见夫人受困,所言退得贼兵者,以小姐妻之,因此愚弟作书请吾兄。

[将军云]既然有此姻缘,可贺,可贺!

[夫人云]安排茶饭者!

[将军云]不索。倘有余党未尽,小官去捕了,却来望贤弟。左右那里,去斩孙飞虎去!

[拿贼了]本欲斩首示众,具表奏闻,见丁文雅失守之罪。恐有未叛者,今将为首各杖一百,余者尽归旧营去者!

[孙飞虎谢了下]

[将军云]张生建退贼之策,夫人面许结亲,若不违前言,淑女可配君子也。

[夫人云]恐小女有辱君子。

[末云]请将军筵席者!

[将军云]我不吃筵席了,我回营去,异日却来庆贺。

[末云]不敢久留兄长,有劳台候。

[将军望蒲关起发]

[众念云]马离普救敲金镫,人望蒲关唱凯歌。[下]

[夫人云]先生大恩,不敢忘也。自今先生休在寺里下,只着仆人寺内养马,足下来家内书院里安歇。我已收拾了,便搬来者。到明日略备草酌,着红娘来请你,是必来一会,别有商议。

[末云]这事都在长老身上。

[问洁云]小子亲事,未如何知?

[洁云]莺莺亲事,拟定妻君。只因兵火至,引起雨云心。[下]

[末云]小子收拾行李,去花园里去也。[下]

\nchapter{第二折}

[夫人上云]今日安排下小酌,单请张生酬劳。道与红娘,疾忙去书院中请张生,着他是必便来,休推故。[下]

[末上云]夜来老夫人说,着红娘来请我,却怎生不见来?我打扮着等他,皂角也使过两个也,水也换了两桶也,乌纱帽擦得光挣挣的,怎么不见红娘来也呵?

[红娘上云]老夫人使我请张生,我想若非张生妙计呵,俺一家儿性命难保也呵!

[中吕][粉蝶儿]半万贼兵,卷浮云片时扫净,俺一家儿死里逃生。舒心的列山灵,陈水陆,张君瑞合当钦敬。当日所望无成,谁想一缄书到为了媒证。

[醉春风]今日个东阁玳筵开,煞强如西厢和月等。薄衾单枕有人温,早则不冷、冷。受用足宝鼎香浓,绣帘风细,绿窗人静。可早来到也。

[脱布衫]幽僻处可有人行?点苍苔白露泠泠。隔窗儿咳嗽了一声。

[红敲门科]

[末云]是谁来也?

[红云]是我。

他启朱唇急来答应。

[末云]拜揖小娘子。

[红唱][小梁州]则见他叉手忙将礼数迎,我这里``万福,先生''。乌纱小帽耀人明,白襕净,角带傲黄程鞓。

[幺篇]衣冠济楚庞儿整,可知道引动俺莺莺。据相貌,凭才性,我从来心硬,一见了也留情。

[末云]既来之,则安之,请书房内说话。小娘子此行为何?

[红云]贱妾奉夫人严命,特请先生小酌数杯,勿却。

[末云]便去,便去。敢问席上有莺莺姐姐么?

[红唱][上小楼]``请''字儿不曾出声,``去''字儿连忙答应;可早莺莺根前,``姐姐''呼之,喏喏连声。秀才每闻道``请'',恰便似听将军严令,和他那五脏神愿随鞭镫。

[末云]今日夫人端的为甚么筵席?

[红唱][幺篇]第一来为压惊,第二来因谢承。不请街坊,不会亲邻,不受人情。避众僧,请老兄,和莺莺匹聘。

[末云]如此小生欢喜。

[红]则见他欢天喜地,谨依来命。

[末云]小生客中无镜,敢烦小娘子,看小生一看何如?

[红唱][满庭芳]来回顾影,文魔秀士,风欠酸丁。下工夫将额颅十分挣,迟和疾擦倒苍蝇,光油油耀花人眼睛,酸溜溜螫得人牙疼。

[末云]夫人办甚么请我?

[红]茶饭已安排定,淘下陈仓米数升,碟下七八碗软蔓青。

[末云]小生想来,自寺中一见了小姐之后,不想今日得成婚姻,岂不为前生分定?

[红云]姻缘非人力所为,天意尔。

[快活三]咱人一事精,百事精;一无成,百无成。世间草木本无情,自古云:地生连理木,水出并头莲,他犹有相兼并。

[朝天子]休道这生,年纪儿后生,恰学害相思病。天生聪俊,打扮素净,奈夜夜成孤另。才子多情,佳人薄幸,兀的不担阁了人性命。

[末云]你姐姐果有信行?

[红唱]谁无一个信行?谁无一个志诚?凭两个今夜亲折证。

我嘱咐你咱:

[四边静]今宵欢庆,软弱莺莺,可曾惯经?你索款款轻轻,灯下交鸳颈。端详可憎,好煞人也无干净。

[末云]小娘子先行,小生收拾书房便来。敢问那里有甚么景致?

[红唱][耍孩儿]俺那里落红满地胭脂冷,休辜负了良辰媚景。夫人遣妾莫消停,请先生勿得推称。俺那里准备着鸳鸯夜月销金帐,孔雀春风软玉屏。乐奏合欢令,有凤箫象板,锦瑟鸾笙。

[末云]小生书剑飘零,无以为财礼,却是怎生?

[红唱][四煞]聘财断不争,婚姻自有成,新婚燕尔安排定。你明博得跨凤乘鸾客,我到晚来卧看牵牛织女星。休傒幸,不要你半丝儿红线,成就了一世儿前程。

[三煞]凭着你灭寇功,举将能,两般儿功效如红定。为甚俺莺娘心下十分顺?都则为君瑞胸中百万兵。越显得文风盛,受用足珠围翠绕,结果了黄卷青灯。

[二煞]夫人只一家,老兄无伴等,为嫌繁冗寻幽静。

[末云]别有甚客人?

[红唱]单请你个有恩有义闲中客,且回避了无是无非窗下僧。夫人的命,道足下莫教推托,和贱妾即便随行。

[末云]小娘子先行,小生随后便来。

[红唱][收尾]先生休作谦,夫人专意等。常言道``恭敬不如从命'',休使得梅香再来请。[下]

[末云]红娘去了,小生拽上书房门者。我比及到得夫人那里,夫人道:``张生,你来了也?饮几杯酒,去卧房内,和莺莺做亲去!''小生到得卧房内,和姐姐解带脱衣,颠鸾倒凤,同谐鱼水之欢,共效于飞之愿。觑他云鬟低坠,星眼微朦,被翻翡翠,袜绣鸳鸯。不知性命何如,且看下回分解。

[笑云]单羡法本好和尚也:只凭说法口,遂却读书心。[下]

\nchapter{第三折}

[夫人排桌子上云]红娘去请张生,如何不见来?

[红见夫人云]张生着红娘先行,随后便来也。

[末上见夫人施礼科]

[夫人云]前日若非先生,焉得有今日。我一家之命,皆先生所活也。聊备小酌,非为报礼,勿嫌轻意。

[末云]``一人有庆,兆民赖之。''此贼之败,皆夫人之福。万一杜将军不至,我辈皆无免死之术。此皆往事,不必挂齿。

[夫人云]将酒来,先生满饮此杯。

[末云]``长者赐,少者不敢辞。''

[末做饮酒科]

[末把夫人酒了]

[夫人云]先生请坐。

[末云]小子侍立座下,尚然越礼,焉敢与夫人对坐?

[夫人云]道不得个``恭敬不如从命''。

[末谢了,坐]

[夫人云]红娘,去唤小姐来,与先生行礼者。

[红朝鬼门道唤云]老夫人后堂待客,请小姐出来哩!

[旦应云]我身子不些不停当,来不得。

[红云]你道请谁哩?

[旦云]请谁?

[红云]请张生哩。

[旦云]若请张生,扶病也索走一遭。

[红发科了]

[旦上]免除崔氏全家祸,尽在张生半纸书。

[双调][五供养]若不是张解元识人多,别一个怎退干戈?排着酒果,列着笙歌。篆烟微,花香细,散满东风帘幕。救了咱全家祸,殷勤呵正礼,钦敬呵当合。

[新水令]恰才向碧纱窗下画了双蛾,拂拭了罗衣上粉香浮涴,只将指尖儿轻轻的贴了钿窝。若不是惊觉人呵,犹压着绣衾卧。

[红云]觑俺姐姐这个脸儿,吹弹得破,张生有福也呵!

[旦唱][幺篇]没查没利谎偻科,你道我宜梳妆的脸儿吹弹得破。

[红云]俺姐姐天生的一个夫人的样儿。

[旦唱]你那里休聒,不当一个信口开合他命福是如何。我做一个夫人也做得过。

[红云]往常两个都害,今日早则喜也!

[旦唱][乔木查]我相思为他,他相思为我,从今后两下里相思都较可。酬贺间礼当酬贺,俺母亲也好心多。

[红云]敢着小姐和张生结亲呵,怎生不做大筵席,会亲戚朋友,安排小酌为何?

[旦云]红娘,你不知夫人意。

[搅筝琶]他怕我是赔钱货,两当一便成合。据着他举将除贼,也消得家缘过活。费了甚一股那,便待要结丝萝!休波,省人情的奶奶忒虑过,恐怕张罗。

[末云]小子更衣咱。

[做撞见旦科]

[旦唱][庆宣和]门儿外,帘儿前,将小脚那。我恰待目转秋波,谁想那识空便的灵心儿早瞧破,唬得我倒躲,倒躲。

[末见旦科]

[夫人云]小姐近前,拜了哥哥者!

[末背云]呀,声息不好了也!

[旦云]呀,俺娘变了卦也!

[红云]这相思又索害也。

[旦唱][雁儿落]荆棘剌怎动那,死没腾无回豁,措支剌不对答,软兀剌难存坐!

[得胜令]谁承望这即即世世老婆婆,着莺莺做妹妹拜哥哥。白茫茫溢起蓝桥水,不邓邓点着袄庙火。碧澄澄清波,扑剌剌将比目鱼分破。急攘攘因何,扢搭地把双眉锁纳合。

[夫人云]红娘看热酒,小姐与哥哥把盏者!

[旦唱][甜水令]我这里粉颈低垂,蛾眉频蹙,芳心无那,俺可甚``相见话偏多''。星眼朦胧,檀口嗟咨,攧窨不过。这席面儿畅好是乌合!

[旦把酒科]

[夫人央科]

[末云]小生量窄。

[旦云]红娘,接了台盏者!

[折桂令]他其实咽不下玉液金波。谁承望月底西厢,变做了梦里南柯。泪眼偷淹,酩子里揾湿香罗。他那里眼倦开软瘫做一垛;我这里手难抬称不起肩窝。病染沉疴,断然难活。则被你送了人呵,当甚么喽啰!

[夫人云]再把一盏者。

[红递盏了]

[红背与旦云]姐姐,这烦恼是怎生是了?

[旦唱][月上海棠]而今烦恼犹闲可,久后思量怎奈何?有意诉衷肠,争奈母亲侧坐。成抛趓,咫尺间如间阔。

[幺篇]一杯闷酒尊前过,低首无言自摧挫。不甚醉颜酡,却早嫌玻璃盏大,从因我,酒上心来较可。

[夫人云]红娘,送小姐卧房里去者。

[旦辞末出科]

[旦云]俺娘好口不应心也呵!

[乔牌儿]老夫人转关儿没定夺,哑谜儿怎猜破;黑阁落甜话儿将人和,请将来着人不快活。

[江儿水]佳人自来多命薄,秀才每从来懦。闷杀没头鹅,撇下陪钱货,下场头那答儿发付我!

[殿前欢]恰才个笑呵呵,都做了江州司马泪痕多。若不是一封书将半万贼兵破,俺一家怎得存活。他不想结姻缘想甚么?到如今难着莫。老夫人谎到天来大,当日成也是凭个母亲,今日败也是凭个萧何。

[离亭宴带歇指煞]从今后玉容寂寞梨花朵,胭脂浅淡樱桃颗,这相思何时是可?昏邓邓黑海来深,白茫茫陆地来厚,碧悠悠青天来阔;太行山般高仰望,东洋海般深思渴。毒害的恁么!俺娘呵,将颤巍巍双头花蕊搓,香馥馥同心缕带割,长搀搀连理琼枝挫。白头娘不负荷,青春女成担阁,将俺那锦片也似前程蹬脱。俺娘把甜句儿落空了他,虚名儿误赚了我。[下]

[末云]小生醉也,告退。夫人根前,欲一言以尽意,未知可否。前者,贼寇相迫,夫人所言,能退贼者,以莺莺妻之。小生挺身而出,作书与杜将军,庶几得免夫人之祸,今日命小生赴宴,将谓有喜庆之期;不知夫人何见,以兄妹之礼相待?小生非图哺啜而来,此事果若不谐,小生即当告退。

[夫人云]先生纵有活我之恩,奈小姐先相国在日,曾许下老身侄儿郑恒。即日有书赴京,唤去了,未见来。如若此子至,其事将如之何?莫若多以金帛相酬,先生拣豪门贵宅之女,别为之求,先生台意若何?

[末云]既然夫人不与,小生何慕金帛之色!却不道``书中有女颜如玉''?则今日便索告辞。

[夫人云]你且住者,今日有酒也。红娘,扶将哥去书房中歇息,到明日咱别有话说。[下]

[红扶末科]

[末念]有分只熬萧寺夜,无缘难遇洞房春。

[红云]张生,少吃一盏却不好?

[末云]我吃甚么来?

[末跪红科]小生为小姐,昼夜忘餐废寝,魂劳梦断,常忽忽如有所失。自寺中一见,隔墙酬和,迎风带月,受无限之苦楚。甫能得成就婚姻,夫人变了卦,使小生智竭思穷,此事几时是了?小娘子,怎生可怜见小生,将此意申与小姐,知小生之心。就小娘子前解下腰间之带,寻个自尽。

[末念]可怜刺股悬梁志,险作离乡背井魂。

[红云]街上好贱柴,烧你个傻角!你休慌,妾当与君谋之。

[末云]计将安在?小生当筑坛拜将。

[红云]妾见先生有囊琴一张,必善于此。俺小姐深慕于琴。今夕妾与小姐同至花园内烧夜香,但听咳嗽为令,先生动操。看小姐听得时,说甚么言语,却将先生之言达知。若有话说,明日妾来回报。这早晚怕夫人寻,我回去也。[下]

\nchapter{第四折}

[末上云]红娘之言,深有意趣。天色晚也,月儿,你早些出来么![焚香了]呀,却早发擂也。呀,却早撞钟也。

[做理琴科]琴呵,小生与足下湖海相随数年,今夜这一场大功,都在你这神品——金徽、玉轸、蛇腹、断纹、峄阳、焦尾、冰弦之上。天那,却怎生借得一阵顺风,将小生这琴声,吹入俺那小姐玉琢成、粉捏就知音的耳朵里去者!

[旦引红上]

[红云]小姐,烧香去来,好明月也呵!

[旦云]事已无成,烧香何济?月儿,你团圆呵,咱却怎生!

[越调][斗鹌鹑]云敛晴空,冰轮乍涌;风扫残红,香阶乱拥;离恨千端,闲愁万种。夫人哪,``靡不有初,鲜克有终。''他做了影儿里的情郎,我做了个画儿里的爱宠。

[紫花儿序]则落得心儿里念想,口儿里闲题,则索向梦儿里相逢。俺娘昨日个大开东阁,我则道怎生般炮凤烹龙,朦胧!可教我``翠袖殷勤捧玉钟'',却不道``主人情重''?则为那兄妹排连,因此上鱼水难同。

[红云]姐姐,你看月阑,明日敢有风也。

[旦云]风月天边有,人间好事无。

[小桃红]人间看波,玉容深锁绣帏中,怕有人搬弄。想嫦娥西没东生有谁共?怨天公,裴航不作游仙梦。这云似我罗帏数重,只恐怕嫦娥心动,因此上围住广寒宫。

[红做咳嗽科]

[末云]来了。[做理琴科]

[旦云]这甚么响?

[红发科]

[旦唱][天净沙]莫不是步摇得宝髻玲珑?莫不是裙拖得环珮玎咚?莫不是铁马儿檐前骤风?莫不是金钩双控,吉丁当敲响帘栊?

[调笑令]莫不是梵王宫,夜撞钟?莫不是疏竹潇潇曲槛中?莫不是牙尺剪刀声相送?莫不是漏声长滴响壶铜?潜身再听在墙角东,原来是近西厢理结丝桐。

[秃厮儿]其声壮,似铁骑刀枪冗冗;其声幽,似落花流水溶溶;其声高,似风清月朗鹤唳空;其声低,似听儿女语,小窗中,喁喁。

[圣药王]他那里思不穷,我这里意已通,娇鸾雏凤失雌雄。他曲未终,我意转浓,争奈伯劳飞燕各西东,尽在不言中。

我近书窗听咱。

[红云]姐姐,你这里听,我瞧夫人,一会便来。

[末云]窗外是有人,已定是小姐,我将弦改过,弹一曲,就歌一篇,名曰《凤求凰》。昔日司马相如得此曲成事,我虽不及相如,愿小姐有文君之意。

[歌曰]有美人兮,见之不忘。一日不见兮,思之如狂。凤飞翩翩兮,四海求凰。无奈佳人兮,不在东墙。张弦代语兮,欲诉衷肠。何时见许兮,慰我彷徨?愿言配德兮,携手相将。不得于飞兮,使我沦亡。

[旦云]是弹得好也呵!其词哀,其意切,凄凄然如鹤唳天。故使妾闻之,不觉泪下。

[麻郎儿]这的是令他人耳聪,诉自己情衷。知音者芳心自懂,感怀者断肠悲痛。

[幺篇]这一篇与本宫始终不同。又不是清夜闻钟,又不是黄鹤醉翁,又不是泣麟悲凤。

[络丝娘]一字字更长漏永,一声声衣宽带松。别恨离愁,变做一弄。张生呵,越教人知重。

[末云]夫人且做忘恩,小姐,你也说谎也呵!

[旦云]你差怨了我。

[东原乐]这的是俺娘的机变,非干是妾身脱空。若由得我呵,乞求得效鸾凤。俺娘无夜无明并女工,我若得些儿闲空,张生呵,怎教你无人处把妾身做诵。

[绵搭絮]疏帘风细,幽室灯清,都则是一层儿红纸,几榥儿疏棂,兀的不是隔着云山几万重!怎得个人来信息通?便做道十二巫峰,他也曾赋高唐来梦中。

[红云]夫人寻小姐哩,咱家去来。

[旦唱][拙鲁速]则见他走将来气冲冲,怎不教人恨匆匆。唬得人来怕恐。早是不曾转动,女孩儿家直恁响喉咙。紧摩弄,索将他拦纵,则恐怕夫人行把我来厮葬送。

[红云]姐姐,则管听琴怎么?张生着我对姐姐说,他回去也。

[旦云]好姐姐呵,是必再着住一程儿。

[红云]再说甚么?

[旦云]你去呵,

[尾]则说道夫人时下有人唧哝,好共歹不着你落空。不问俺口不应的狠毒娘,怎肯着别离了志诚种。

[并下]

[络丝娘煞尾]不争惹恨牵情斗引,少不得废寝忘餐病症。

\newpage

题目:张君瑞破贼计\ 莽和尚生杀心

正名:小红娘昼请客\ 崔莺莺夜听琴

\npart{第三本\ 张君瑞害相思杂剧}

\nchapter{楔子}

[旦上云]自那夜听琴后,闻说张生有病,我如今着红娘去书院里,看他说甚么。[叫红科]

[红上云]姐姐唤我,不知有甚事,须索走一遭。

[旦云]这般身子不快呵,你怎么不来看我?

[红云]你想张……

[旦云]张甚么?

[红云]我张着姐姐哩。

[旦云]我有一件事,央及你咱。

[红云]甚么事?

[旦云]你与我望张生走一遭,看他说甚么,你来回我话者。

[红云]我不去,夫人知道不是耍。

[旦云]好姐姐,我拜你两拜,你便与我走一遭。

[红云]侍长请起,我去则便了。说道:``张生,你好生病重,则俺姐姐也不弱。''只因午夜调琴手,引起春闺爱月心。

[仙吕][赏花时]俺姐姐针线无心不待拈,脂粉香消懒去添,春恨压眉尖。若得灵犀一点,敢医可了病恹恹。[下]

[旦云]红娘去了,看他回来说甚话,我自有主意。[下]

\nchapter{第一折}

[末上云]害杀小生也。自那夜听琴后,再不能够见俺那小姐。我着长老说将去,道:``张生好生病重!''却怎生不见人来看我?却思量上来,我睡些儿咱。

[红上云]奉小姐言语,着我看张生,须索走一遭。我想咱每一家,若非张生,怎存俺一家儿性命也!

[仙吕][点绛唇]相国行祠,寄居萧寺。因丧事,幼女弧儿,将欲从军死。

[混江龙]谢张生伸志,一封书到便兴师。显得文章有用,足见天地无私。若不是剪草除根半万贼,险些儿灭门绝户了俺一家儿。莺莺君瑞,许配雄雌;夫人失信,推托别词;将婚姻打灭,以兄妹为之。如今都废却成亲事。一个价糊突了胸中锦绣,一个价泪揾湿了脸上胭脂。

[油葫芦]憔悴潘郎鬓有丝,杜韦娘不似旧时,带围宽清减了瘦腰肢。一个睡昏昏不待观经史,一个意悬悬懒去拈针指;一个丝桐上调弄出离恨谱,一个花笺上删抹成断肠诗;一个笔下写幽情,一个弦上传心事:两下里都一样害相思。

[天下乐]方信道才子佳人信有之,红娘看时,有些乖性儿,则怕有情人不遂心也似此。他害的有些抹媚,我遭着没三思,一纳头安排着憔悴死。却早来到书院里,我把唾津儿润破窗纸,看他在书房里做甚么。

[村里迓鼓]我将这纸窗儿湿破,悄声儿窥视。多管是和衣儿睡起,罗衫上前襟褶衤至。孤眠况味,凄凉情绪,无人伏侍。觑了他涩滞气色,听了他微弱声场息,看了他黄瘦脸儿。张生呵,你若不闷死,多应是害死。

[元和令]金钗敲门扇儿。

[末云]是谁?

[红唱]我是个散相思的五瘟使。俺小姐想着风清月朗夜深时,使红娘来探尔。

[末云]既然小娘子来,小姐必有言语。

[红唱]俺小姐至今脂粉未曾施,念到有一千番张殿试。

[末云]小姐既有见怜之心,小生有一简,敢烦小娘子达知肺腑咱。

[红云]只恐他翻了面皮。

[上马娇]他若是见了这诗,看了这词,他敢颠倒费神思。他拽紥起面皮来:``查得谁的言语你将来,这妮子怎敢胡行事!''他可敢嗤嗤的扯做了纸条儿。

[末云]小生久后多以金帛拜酬小娘子。

[红唱][胜葫芦]哎,你个馋穷酸俫没意儿,卖弄你有家私,莫不图谋你的东西来到此?先生的钱物,与红娘做赏赐,是我爱你的金资?

[幺篇]你看人似桃李春风墙外枝,卖俏倚门儿。我虽是个婆娘有气志,则说道:``可怜见小子只身独自!''恁的呵,颠倒有个寻思。

[末云]依着姐姐,可怜见小子只身独自!

[红云]兀的不是也,你写来,咱与你将去。

[末写科]

[红云]写得好呵,读与我听咱。

[末读云]``珙百拜,奉书芳卿可人妆次:自别颜范,鸿稀鳞绝,悲怆不胜。孰料夫人以恩成怨,变易前姻,岂得不为失信乎?使小生目视东墙,恨不得腋翅于汝台左右;患成思渴,垂命有日。因红娘至,聊奉数字,以表寸心。万一有见怜之意,书以掷下,庶几尚可保养。造次不谨,伏乞情恕。后成五言诗一首,就书录呈:`相思恨转添,谩把瑶琴弄。乐事又逢春,芳心尔亦动。此情不可违,虚誉何须奉。莫负月华明,且怜花影重。' ''

[红唱][后庭花]我则道拂花笺打稿儿,原来他染霜毫不勾思。先写下几句寒温序,后题着五言八句诗。不移时,把花笺锦字,叠做个同心方胜儿。忒聪明,忒敬思,忒风流,忒浪子。虽然是假意儿,小可的难到此。

[青歌儿]颠倒写鸳鸯两字,方信道``在心为志''。

[末云]姐姐将去,是必在意者!

[红唱]看喜怒其间觑个意儿。放心波学士!我愿为之,并不推辞,自有言词。则说道:``昨夜弹琴的那人儿,教传示。''这简帖儿我与你将去,先生当以功名为念,休堕了志气者!

[寄生草]你将那偷香手,准备着折桂枝。休教那淫词儿污了龙蛇字,藕丝儿缚定鹍鹏翅,黄莺儿夺了鸿鹄志;休为这悴帏锦帐一佳人,误了你玉堂金马三学士。

[末云]姐姐在意者!

[红云]放心,放心。

[煞尾]沈约病多般,宋玉愁无二,清减了相思样子。则你那眉眼传情未了时,我中心日夜藏之。怎敢因而,``有美玉于斯'',我须教有发落归着这张纸。凭着我舌尖上说词,更和这简帖儿里心事,管教那人儿来探你一遭儿。[下]

[末云]小娘子将简帖儿去了,不是小生说口,则是一道会亲的符箓。他明日回话,必有个次第。且放下心,须索好音来也。``且将宋玉风流策,寄与蒲东窈窕娘。''[下]

\nchapter{第二折}

[旦上云]红娘伏侍老夫人,不得空,偌早晚敢待来也。困思上来,再睡些儿咱。[睡科]

[红上云]奉小姐言语,去看张生,因伏侍老夫人,未曾回小姐话去。不听得声音,敢又睡哩,我入去看一遭。

[中吕][粉蝶儿]风静帘闲,透纱窗麝兰香散,启朱扉摇响双环。绛台高,金荷小,银釭犹灿。比及将暖帐轻弹,先揭起这梅红罗软帘偷看。

[醉春风]则见他钗亸玉横斜,髻偏云乱挽。日高犹自不明眸,畅好是懒、懒。

[旦做起身长叹科]

[红唱]半晌抬身,几回搔耳,一声长叹。我待便将简帖儿与他,恐俺小姐有许多假处哩。我则将这简帖儿放在妆盒儿上,看他见了说甚么。

[旦做照镜科,见帖看科]

[红唱][普天乐]晚妆残,乌云亸,轻匀了粉脸,乱挽起云鬟。将简帖儿拈,把妆盒儿按,开拆封皮孜孜看,颠来倒去不害心烦。

[旦怒叫]红娘!

[红做意云]呀,决撒了也!厌的早扢皱了黛眉。

[旦云]小贱人,不来怎么!

[红唱]忽的波低垂了粉颈,氲的呵改变了朱颜。

[旦云]小贱人,这东西那里将来的?我是相国的小姐,谁敢将这简帖来戏弄我?我几曾惯看这等东西?告过夫人,打下你个小贱人下截来。

[红云]小姐使将我去,他着我将来,我不识字,知他写着甚么?

[快活三]分明是你过犯,没来由把我摧残;使别人颠倒恶心烦。你不``惯'',谁曾``惯''?姐姐休闹,比及你对夫人说呵,我将这简帖儿,去夫人行出首去来!

[旦做揪住科]我逗你耍来。

[红云]放手,看打下下截来!

[旦云]张生两日如何?

[红云]我则不说。

[旦云]好姐姐,你说与我听咱!

[红唱][朝天子]张生近间面颜,瘦得来实难看。不思量茶饭,怕见动弹;晓夜将佳期盼,废寝忘餐。黄昏清旦,望东墙淹泪眼。

[旦云]请个好太医看他证候咱。

[红云]他证候吃药不济。病患要安,则除是出几点风流汗。

[旦云]红娘,不看你面时,我将与老夫人看,看他有何面目见夫人!虽然我家亏他,只是兄妹之情,焉有外事。红娘,早是你口稳哩,若别人知呵,甚么模样!

[红云]你哄着谁哩!你把这个饿鬼,弄得他七死八活,却要怎么?

[四边静]怕人家调犯,``早共晚夫人见些破绽,你我何安。''问甚么他遭危难?撺断得上竿,掇了梯儿看。

[旦云]将描笔儿过来,我写将去回他,着他下次休是这般!

[旦做写科]

[起身科云]红娘,你将去说:``小姐看望先生,相待兄妹之礼如此,非有他意。再一遭儿是这般呵,必告夫人知道。''和你个小贱人都有话说!

[旦掷书下]

[红唱][脱布衫]小孩儿家口没遮拦,一迷的将言语摧残。把似你使性子,休思量秀才,做多少好人家风范。

[红做拾书科][小梁州]他为你梦里成双觉后单,废寝忘餐。罗衣不奈五更寒,愁无限,寂寞泪阑干。

[幺篇]似这等辰勾空把佳期盼,我将这角门儿世不曾牢拴,则愿你做夫妻无危难。我向这筵席头上整扮,做一个缝了口的撮合山。

[红云]我若不去来,道我违拗他,那生又等我回报,我须索走一遭。[下]

[末上云]那书倩红娘将去,未见回话。我这封书去,必定成事。这早晚敢侍来也。

[红上]须索回张生话去。小姐,你性儿忒惯得娇了!有前日的心,那得今日的心来?

[石榴花]当日个晚妆楼上杏花残,犹自怯衣单;那一片听琴心清露月明间。昨日个向晚,不怕春寒,几乎险被先生馔。那其间岂不胡颜?为一个不酸不醋风魔汉,隔墙儿险化做了望夫山。

[斗鹌鹑]你用心儿拨雨撩云,我好意儿传书寄简。不肯搜自己狂为,则待要觅别人破绽。受艾焙权时忍这番,畅好是奸!``张生是兄妹之礼,焉敢如此!''对人前巧语花言;没人处便想张生,背地里愁眉泪眼。

[红见末科]

[末云]小娘子来了,擎天柱,大事如何了也?

[红云]不济事了,先生休傻。

[末云]小生简帖儿,是一道会亲的符篆,则是小娘子不用心,故意如此。

[红云]我不用心?有天理!你那简帖儿好听!

[上小楼]这的是先生命慳,须不是红娘违慢。那简帖儿倒做了你的招状,他的勾头,我的公案。若不是觑面颜,厮顾盼,担饶轻慢。先生受罪,礼之当然。贱妾何辜?争些儿把你娘拖犯!

[幺篇]从今后相会少,见面难。月暗西厢,凤去秦楼,云敛巫山。你也赸,我也赸,请先生休讪,早寻个酒阑人散。

[红云]只此再不必申诉足下肺腑,怕夫人寻,我回去也。

[末云]小娘子此一遭去,再着谁与小生分剖?必索做一个道理,方可救得小生一命。[末跪下揪住红科]

[红云]张先生是读书人,岂不知此意,其事可知矣。

[满庭芳]你休要呆里撒奸。你待要恩情美满,却教我骨肉摧残。老夫人手执着棍儿摩娑看,粗麻线怎透得针关?直待我拄着拐帮闲钻懒,缝合唇送暖偷寒。待去呵,小姐性儿撮盐入火,消息儿踏着泛;待不去呵,

[末跪哭云]小生这一个性命,都在小娘子身上。

[红唱]禁不得你甜话儿热趱。好着我两下里难人做。我没来由分说,小姐回与你的书,你自看者。

[末接科,开读科]呀,有这场喜事!撮土焚香,三拜礼毕。早知小姐简至,理合远接;接待不及,勿令见罪。小娘子,和你也欢喜。

[红云]怎么?

[末云]小姐骂我都是假,书中之意,着我今夜花园里来,和他``哩也波,哩也罗''哩!

[红云]你读书我听。

[末云]``待月西厢下,迎风户半开。隔墙花影动,疑是玉人来。''

[红云]怎见得他着你来?你解与我听咱。

[末云]``待月西厢下'',着我月上来;``迎风户半开'',他开门待我;``隔墙花影动,疑是玉人来'',着我跳过墙来。

[红笑云]他着你跳过墙来,你做下来。端的有此说么?

[末云]俺是个猜诗谜的社家,风流隋河,浪子陆贾。我那里有差的勾当?

[红云]你看我姐姐,在我行也使这般道儿。

[耍孩儿]几曾见寄书的颠倒瞒着鱼雁,小则小心肠儿转关。写着道西厢待月等得更阑,着你跳东墙``女''字边``干''。原来那诗句儿里包笼着三更枣,简帖儿里埋伏着九里山。他着紧处将人慢。恁会云雨闹中取静,我寄音书忙里偷闲。

[四煞]纸光明玉板,字香喷麝兰,行儿边湮透非春汗?一缄情泪红犹湿,满纸春愁墨未干。从今后休疑难,放心波玉堂学士,稳情取金雀鸦鬟。

[三煞]他人行别样的亲,俺根前取次看,更做道孟光接了梁鸿案。别人行甜言美语三冬暖,我根前恶语伤人六月寒。我为头儿看:看你个离魂倩女,怎发付掷果潘安。

[末云]小生读书人,怎跳得那花园过也。

[红唱][二煞]隔墙花又低,迎风户半拴,偷香手段今番按。怕墙高怎把龙门跳?嫌花密难将仙桂攀。放心去,休辞惮。你若不去呵,望穿他盈盈秋水,蹙损了淡淡春山。

[末云]小生曾到那花园里,已经两遭,不见那好处。这一遭,知他又怎么?

[红云]如今不比往常。

[煞尾]你虽是去了两遭,我敢道不如这番。你那隔墙酬和都胡侃,证果的是今番这一简。[红下]

[末云]万事自有分定,谁想小姐有此一场好处。小生是猜诗谜的社家,风流隋何,浪子陆贾,到那里扢紥帮便倒地。今日颓天百般的难得晚。天,你有万物于人,何故争此一日?疾下去波!读书继晷怕黄昏,不觉西沉强掩门。欲赴海棠花下约,太阳何苦又生根?[看天云]呀,才晌午也,再等一等。[又看科]今日万般的难得下去也呵!碧天万里无云,空劳倦客身心。恨杀鲁阳贪战,不教红日西沉。呀,却早倒西也,再等一等咱。无端三足乌,团团光烁烁。安得后羿弓,射此一轮落!谢天地,却早日下去也。呀,却早发擂也!呀,却早撞钟也!拽上书房门,到得那里,手挽着垂杨,滴流扑跳过墙去。[下]

\nchapter{第三折}

[红上云]今日小姐着我寄书与张生,当面偌多般意儿,原来诗内暗约着他来。小姐也不对我说,我也不瞧破他,则请他烧香。今夜晚妆处比每日较别,我看他到其间怎的瞒我?

[红唤科]姐姐,咱烧香去来。

[旦上云]花阴重叠香风细,庭院深沉淡月明。

[红云]今夜月明风清,好一派景致也呵!

[双调][新水令]晚风寒峭透窗纱,控金钩绣帘不挂。门阑凝暮霭,楼角敛残霞。恰对菱花,楼上晚妆罢。

[驻马听]不近喧哗,嫩绿池溏藏睡鸭;自然幽雅,淡黄杨柳带栖鸦。金莲蹴损牡丹芽,玉簪抓住荼蘼架。夜凉苔径滑,露珠儿湿透了凌波袜。我看那生和俺小姐巴不得到晚。

[乔牌儿]自从那日初时想月华,捱一刻似一夏。见柳梢斜日迟迟下,早道``好教贤圣打''。

[搅筝琶]打扮的身子儿诈,准备着云雨会巫峡。只为这燕侣莺俦,锁不住心猿意马。不则俺那姐姐害,那生呵——二三日来水米不粘牙。因姐姐闭月羞花,真假,这其间性儿难按纳,一地里胡拿。姐姐这湖山下立地,我开了寺里角门儿。怕有人听俺说话,我且看一看。

[做意了]偌早晚傻角却不来``赫赫赤赤''来?

[末云]这其间正好去也,赫赫赤赤。

[红云]那鸟来了。

[沉醉东风]我则道槐影风摇暮鸦,原来是玉人帽侧乌纱。一个潜身在曲槛边,一个背立在湖山下。那里叙寒温?并不曾打话。

[红云]赫赫赤赤,那鸟来了。

[末云]小姐,你来也。[搂住红科]

[红云]禽兽!

[末云]是我。

[红云]你看得好仔细着!若是夫人怎了?

[末云]小生害得眼花,搂得慌了些儿,不知是谁。望乞恕罪。

[红唱]便做道搂得慌呵,你好索觑咱,多管是饿得你个穷神眼花。

[末云]小姐在那里?

[红云]在湖山下。我问你咱:真个着你来哩?

[末云]小生猜诗谜社家,风流隋何,浪子陆贾,准定扢紥帮便倒地。

[红云]你休从门里去,则道我使你来。你跳过这墙去,今夜这一弄儿助你两个成亲。我说与你,依着我者。

[乔牌儿]你看那淡云笼月华,似红纸护银蜡;柳丝花朵垂帘下,绿莎茵铺着绣榻。

[甜水令]良夜迢迢,闲庭寂静,花枝低亚。他是个女孩儿家,你索将性儿温存,话儿摩弄,意儿谦洽。休猜做败柳残花。

[折桂令]他是个娇滴滴美玉无瑕,粉脸生春,云鬓堆鸦。恁的般受怕担惊,又不图甚浪酒闲茶。则你那夹被儿时当奋发,指头儿告了消乏。打叠起嗟呀,毕罢了牵挂,收拾了忧愁,准备着撑达。

[末作跳墙搂旦科]

[旦云]是谁?

[末云]是小生。

[旦怒云]张生,你是何等之人!我在这里烧香,你无故至此。若夫人闻知,有何理说?

[末云]呀,变了卦也!

[红唱][锦上花]为甚媒人,心无惊怕?赤紧的夫妻每、意不争差。我这里蹑足潜踪,悄地听咱:一个羞惭,一个怒发。

[幺篇]张生无一言,呀,莺莺变了卦。一个悄悄冥冥,一个絮絮答答。却早禁住隋何,迸住陆贾,叉手躬身,妆聋做哑。张生背地里嘴那里去了?向前搂住丢番,告到官司,怕羞了你?

[清江引]没人处则会闲嗑牙,就里空奸诈。怎想湖山边,不记``西厢下''。香美娘处分破花木瓜。

[旦云]红娘,有贼!

[红云]是谁?

[末云]是小生。

[红云]张生,你来这里有甚么勾当?

[旦云]扯到夫人那里去。

[红云]到夫人那里,怕坏了他行止。我与姐姐处分他一场。张生,你过来,跪着!

[生跪科]

[红云]你既读孔圣之书,必达周公之礼。夤夜来此何干?

[雁儿落]不是俺一家儿乔作衙,说几句衷肠话:我则道你文学海样深,谁知你色胆有天来大。

[红云]你知罪么?

[末云]小生不知罪。

[红唱][得胜令]谁着你夤夜入人家,非奸做贼拿。你本是个折桂客,做了偷花汉;不想去跳龙门,学骗马。姐姐,且看红娘面,饶过这生者。

[旦云]若不看红娘面,扯你到夫人那里去,看你有何面目见江东父老!起来。

[红唱]谢小姐贤达,看我面遂情罢。若到官司详察,``你既是秀才,只合苦志于寒窗之下,谁教你夤夜辄入人家花园?做得个非奸即盗。''先生呵,整备着皮肤吃顿打。

[旦云]先生虽有活人之恩,恩则当报。既为兄妹,何生此心?万一夫人知之,先生何以自安?今后再勿如此。若更为之,与足下决无干休![下]

[末朝鬼门道云]你着我来,却怎么有偌多说话?

[红扳过末云]羞也,羞也!却不``风流隋何,浪子陆贾''?

[末云]得罪波``社家'',今日便早则死心塌地。

[红唱][离亭宴带歇指煞]再休题春宵一刻千金价,准备着寒窗更守十年寡。猜诗谜的社家, 拍了``迎风户半开'',山障了``隔墙花影动'',绿惨了``待月西厢下''。你将何郎粉面搽,他自把张敞眉儿画。强风情措大。晴干了尤云殢雨心,悔过了窃玉偷香胆,删抹了倚翠偎红话。

[末云]小生再写一简,烦小娘子将去,以尽衷情如何?

[红唱]淫词儿早则休,简帖儿从今罢。犹古自参不透风流调法。从今后悔罪也卓文君,你与我学去波汉司马。[下]

[末云]你这小姐送了人也!此一念小生再不敢举。奈有病体日笃,将如之奈何?夜来得简方喜,今日强扶至此,又值这一场怨气,眼见休也。则索回书房中纳闷去。桂子闲中落,槐花病里看。[下]

\nchapter{第四折}

[夫人上云]早间长老使人来,说张生病重。我着长老使人请个太医去看了,一壁道与红娘,看哥哥行问汤药去者。问太医下甚么药,证候如何,便来回话。[下]

[红上云]老夫人才说张生病沉重,昨晚吃我那一场气,越重了。莺莺呵,你送了他人。[下]

[旦上云]我写一简,则说道药方,着红娘将去与他,证候便可。[旦唤红科]

[红云]姐姐,唤红娘怎么?

[旦云]张生病重,我有一个好药方儿,与我将去咱。

[红云]又来也。娘呵,休送了他人!

[旦云]好姐姐,救人一命,将去咱。

[红云]不是你,一世也救他不得!如今老夫人使我去哩,我就与你将去走一遭。[下]

[旦云]红娘去了,我绣房里等他回话。[下]

[末上云]自从昨夜花园中吃了这一场气,投着旧证候,眼见得休了也。老夫人说,着长老唤太医来看我;我这颓证候,非是太医所治的。则除是那小姐美甘甘、香喷喷、凉渗渗、娇滴滴一点儿唾津儿咽下去,这屌病便可。

[洁引太医上,``双斗医''科范了][下]

[洁云]下了药了,我回夫人话去,少刻再来相望。[下]

[红上云]俺小姐送得人如此,又着我去动问,送药方儿去,越着他病沉了也。我索走一遭。异乡易得离愁病,妙药难医肠断人!

[越调][斗鹌鹑]则为你彩笔题诗,回文织锦;送得人卧枕着床,忘餐废寝;折倒得鬓似愁潘,腰如病沈。恨已深,病已沉,昨夜个热脸儿对面抢白,今日个冷句儿将人厮侵。昨夜这般抢白他呵!

[紫花儿序]把似你休倚着栊门儿待月,依着韵脚儿联诗,侧着耳朵儿听琴。见了他撇假偌多话:``张生,我与你兄妹之礼,甚么勾当!''怒时节把一个书生来跌噷。欢时节:``红娘,好姐姐,去望他一遭!''将一个侍妾来逼临。难禁,好着我似线脚儿般殷勤不离了针。从今后教他一任。这的是俺老夫人的不是——将人的义海恩山,都做了远水遥岑。

[红见末问云]哥哥病体若何?

[末云]害杀小生也!我若是死呵,小娘子,阎王殿前少不得你做个干连人。

[红叹云]普天下害相思的,不似你这个傻角。

[天净纱]心不存学海文林,梦不离柳影花阴,则去那窃玉偷香上用心。又不曾得甚,自从海棠开想到如今。因甚的便病得这般了?

[末云]都因你行——怕说的谎——因小侍长上来!当夜书房一气一个死。小生救了人,反被害了。自古人云:``痴心女子负心汉'',今日反其事了。

[红唱][调笑令]我这里自审,这病为邪淫,尸骨嵒嵒鬼病侵。更做道秀才每从来恁。似这般干相思的好撒㖔。功名上早则不遂心,婚姻上更返吟复吟。

[红云]老夫人着我来,看哥哥要甚么汤药。小姐再三伸敬,有一药方,送来与先生。

[末做慌科]在那里?

[红云]用着几般儿生药,各有制度,我说与你:

[小桃红]``桂花''摇影夜深沉,酸醋``当归''浸。

[末云]桂花性温,当归活血,怎生制度?

[红唱]面靠着湖山背阴里窨。这方儿最难寻,一服两服令人恁。

[末云]忌甚么物?

[红唱]忌的是``知母''未寝,怕的是``红娘''撒沁。吃了呵,稳情取``使君子''一星儿``参''。这药方儿,小姐亲笔写的。

[末看药方大笑科]

[末云]早知姐姐书来,只合远接,小娘子……

[红云]又怎么?却早两遭也。

[末云]不知这首诗意,小姐待和小生``里也波''哩。

[红云]不少了一些儿?

[鬼三台]足下其实啉,休装㖔。笑你个风魔的翰林,无处问佳音,向简帖儿上计禀。得了个纸条儿恁般绵里针,若见玉天仙怎生软厮禁?俺那小姐忘恩,赤紧的偻人负心。书上如何说?你读与我听咱。

[末念云]``休将闲事苦萦怀,取次摧残天赋才。不意当时完妾命,岂防今日作君灾?仰图厚德难从礼,谨奉新诗可当谋。寄与高唐休咏赋,今宵端的雨云来。''此韵非前日之比,小姐必来。

[红云]他来呵,怎生?

[秃厮儿]身卧着一条布衾,头枕着三尺瑶琴;他来时怎生和你一处寝?冻得来战兢兢,说甚知音?

[圣药王]果若你有心,他有心,昨日秋千院宇深沉;花有阴,月有阴,``春宵一刻抵千金'',何须``诗对会家吟''?

[末云]小生有花银十两,有铺盖赁与小生一付。

[红唱][东原乐]俺那鸳鸯枕,翡翠衾,便遂杀了人心,如何肯赁?至如你不脱解和衣儿更怕甚?不强如手执定指尖儿恁?倘或成亲,到大来福荫。

[末云]小生为小姐如此容色,莫不小姐为小生也减动丰韵么?

[红唱][绵搭絮]他眉弯远山不翠,眼横秋水无光,体若凝酥,腰如弱柳,俊的是庞儿俏的是心,体态温柔性格儿沉。虽不会法灸神针,更胜似救苦难观世音。

[末云]今夜成了事,小生不敢有忘。

[红唱][幺篇]你口儿里漫沉吟,梦儿里苦追寻。往事已沉,只言目今,今夜相逢管教恁。不图你甚白壁黄金,则要你满头花,拖地锦。

[末云]怕夫人拘系,不能够出来。

[红云]则怕小姐不肯。果有意呵,

[煞尾]虽然是老夫人晓夜将门禁,好共歹须教你称心。

[末云]休似昨夜不肯。

[红云]你挣揣咱。来时节肯不肯尽由他,见时节亲不亲在于您。

[并下]

[络丝娘煞尾]因今宵传言送语,看明日携云握雨。

\newpage

题目:老夫人命医士\ 崔莺莺寄情诗

正名:小红娘问汤药\ 张君瑞害相思

\npart{第四本草桥店梦莺莺杂剧}

\nchapter{楔子}

[旦上云]昨夜红娘传简去与张生,约今夕和他相见,等红娘来做个商量。

[红上云]姐姐着我传简帖儿与张生,约他今宵赴约。俺那小姐,我怕又有说谎,送了他性命,不是耍处。我见小姐,看他说甚么。

[旦云]红娘,收拾卧房,我睡去。

[红云]不争你要睡呵,那里发付那生?

[旦云]甚么那生?

[红云]姐姐,你又来也,送了人性命,不是耍处!你若又番悔,我出首与夫人:你着我将简帖儿约下他来。

[旦云]这小贱人倒会放刁。羞人答答的,怎生去!

[红云]有甚的羞?到那里则合着眼者!

[红催莺云]去来,去来!老夫人睡了也。

[旦走科]

[红云]俺姐姐语言虽是强,脚步儿早先行也。

[仙吕][端正好]因姐姐玉精神,花模样,无倒断晓夜思量。着一片志诚心,盖抹了漫天谎。出画阁,向书房,离楚岫,赴高唐,学窃玉,试偷香,巫娥女,楚襄王。楚襄王敢先在阳台上。[下]

\nchapter{第一折}

[末上云]昨夜红娘所遗之简,约小生今夜成就。这早晚初更尽也,不见来呵,小姐休说谎咱!人间良夜静复静,天上美人来不来?

[仙吕][点绛唇]伫立闲阶,夜深香霭、横金界。潇洒书斋,闷杀读书客。

[混江龙]彩云何在?月明如水浸楼台。僧归禅室,鸦噪庭槐。风弄竹声、则道似金珮响?月移花影,疑是玉人来。意悬悬业眼,急攘攘情怀,身心一片,无处安排,则索呆答孩倚定门儿待。越越的表鸾信杳,黄犬音乖。

小生一日十二时,无一刻放下小姐。你那里知道呵!

[油葫芦]情思昏昏眼倦开,单枕侧,梦魂飞入楚阳台。早知道无明夜因他害,想当初不如不遇倾城色。人有过,必自责,勿惮改。我却待``贤贤易色''将心戒,怎禁他兜的上心来。

[天下乐]我则索倚定门儿手托腮,好着我难猜:来也那不来?夫人行料应难离侧。望得人眼欲穿,想得人心越窄,多管是冤家不自在。

偌早晚不来,莫不又是谎么?

[那吒令]他若是肯来,早身离贵宅;他若是到来,便春生敝斋;他若是不来,似石沉大海。数着他脚步儿行,倚定窗棂儿待。寄语多才:

[鹊踏枝]恁的般恶抢白,并不曾记心怀;拨得个意转心回,夜去明来。空调眼色经今半载,这其间委实难捱。

小姐这一遭若不来呵——

[寄生草]安排着害,准备着抬。想着这异乡身强把茶汤捱,则为这可憎才熬得心肠耐,办一片志诚心留得形骸在。试着那司天台打算半年愁,端的是太平车约有十余载。

[红上云]姐姐,我过去,你在这里。[红敲科]

[末问云]是谁?

[红云]是你前世的娘。

[末云]小姐来么?

[红云]你接了衾枕者,小姐入来也。张生,你怎么谢我?

[末拜云]小生一言难尽,寸心相报,惟天可表!

[红云]你放轻者,休唬了他。

[红推旦入云]姐姐,你入去,我在门儿外等你。

[末见旦跪云]张生有何德能,敢劳神仙下降,知他是睡里梦里?

[村里迓鼓]猛见他可憎模样,小生那里病来?早医可九分不快。先前见责,谁承望今宵欢爱!着小姐这般用心,不才张珙,合当跪拜。小生无宋玉般容,潘安般貌,子建般才。姐姐,你则是可怜见为人在客。

[元和令]绣鞋儿刚半拆,柳腰儿够一搦,羞答答不肯把头抬,只将鸳枕捱。云鬟仿佛坠金钗,偏宜䯼髻儿歪。

[上马娇]我将这钮扣儿松,把缕带儿解,兰麝散幽斋。不良会把人禁害,咍,怎不肯回过脸儿来?

[胜葫芦]我这里软玉温香抱满怀。呀,阮肇到天台。春至人间花弄色,将柳腰款摆,花心轻拆,露滴牡丹开。

[幺篇]但蘸着些儿麻上来,鱼水得和谐,嫩蕊娇香蝶恣采。半推半就,又惊又爱,檀口揾香腮。

[末跪云]谢小姐不弃,张珙今夕得就枕席,异日犬马之报。

[旦云]妾千金之躯,一旦弃之。此身皆托于足下,勿以他日见弃,使妾有白头之叹。

[末云]小生焉敢如此!

[末看手帕科][后庭花]春罗原莹白,早见红香点嫩色。

[旦云]羞人答答的,看甚么。

[末唱]灯下偷睛觑,胸前着肉揣。畅奇哉!浑身通泰,不知春从何处来。无能的张秀才,孤身西洛客,自从逢稔色,思量的不下怀。忧愁因间隔,相思无摆划。谢芳卿不见责。

[柳叶儿]我将你做心肝儿般看待,点污了小姐清白。忘餐废寝舒心害,若不是真心耐,志诚捱,怎能够这相思苦尽甘来?

[青哥儿]成就了今宵欢爱,魂飞在九霄云外。投至得见你多情小奶奶,憔悴形骸,瘦似麻秸。今夜和谐,犹自疑猜。露滴香埃,风静闲阶,月射书斋,云锁阳台。审问明白,只疑是昨夜梦中来,愁无奈。

[旦云]我回去也,怕夫人觉来寻我。

[末云]我送小姐出来。

[寄生草]多丰韵,忒稔色:乍时相见教人害,霎时不见教人怪,些儿得见教人爱。今宵同会碧纱厨,何时重解香罗带?

[红云]来拜你娘!张生,你喜也!姐姐,咱家去来。

[末唱][煞尾]春意透酥胸,春色横眉黛,贱却人间玉帛。杏脸桃腮,乘着月色,娇滴滴越显得红白。下香阶,懒步苍苔,动人处弓鞋凤头窄。叹鲰生不才,谢多娇错爱。若小姐不弃小生,此情一心者,你是必破工夫明夜早些来。[下]

\nchapter{第二折}

[夫人引俫上云]这几日窃见莺莺语言恍惚,神思加倍,腰肢体态,比向日不同。莫不做下来了么?

[俫云]前日晚夕,奶奶睡了,我见姐姐和红娘烧香,半晌不回来,我家去睡了。

[夫人云]这桩事都在红娘身上。唤红娘来!

[俫唤红科]

[红云]哥哥唤我怎么?

[俫云]奶奶知道你和姐姐去花园里去,如今要打你哩。

[红云]呀,小姐,你带累我也!小哥哥你先去,我便来也。

[红唤旦科]姐姐,事发了也。老夫人唤我哩,却怎了?

[旦云]好姐姐,遮盖咱!

[红云]娘呵,你做的隐秀者——我道你做下来也!

[旦念]月圆便有阴云蔽,花发须教急雨催。

[红唱][越调][斗鹌鹑]则着你夜去明来,倒有个天长地久;不争你握雨携云,常使我提心在口。则合带月披星,谁着你停眠整宿?老夫人心数多,情性㑇,使不着我巧语花言,将没做有。

[紫花儿序]老夫人猜那穷酸做了新婿,小姐做了娇妻,``这小贱人做了牵头''。俺小姐这些时春山低翠,秋水凝眸。别样的都休,试把你裙带儿拴,纽门儿扣,比着你旧时肥瘦,出落得精神,别样的风流。

[旦云]红娘,你到那里,小心回话者。

[红云]我到夫人处,必问:``这小贱人!

[金蕉叶]我着你但去处行监坐守,谁着你迤逗的胡行乱走?''若问着此一节呵如何诉休?你便索与他个知情的犯由。姐姐,你受责理当,我图甚么来?

[调笑令]你绣帏里效绸缪,倒凤颠鸾百事有。我在窗儿外几曾轻咳嗽,立苍苔将绣鞋儿冰透。今日个嫩皮肤倒将粗棍抽,姐姐呵,俺这通殷勤的着甚来由?姐姐在这里等着,我过去。说过呵,休欢喜;说不过,休烦恼。

[红见夫人科]

[夫人云]小贱人,为甚么不跪下!你知罪么?

[红跪云]红娘不知罪。

[夫人云]你故自口强哩。若实说呵,饶你;若不实说呵,我直打死你这个贱人!谁着你和小姐花园里去来?

[红云]不曾去,谁见来?

[夫人云]欢郎见你去来,尚故自推哩!

[打科]

[红云]夫人,休闪了手。且息怒停嗔,听红娘说。

[鬼三台]夜坐时停了针绣,共姐姐闲穷究,说张生哥哥病久,咱两个背着夫人向书房问候。

[夫人云]问候呵,他说甚么?

[红云]他说来,道``老夫人事已休,将恩变为仇,着小生半途喜变做忧。''他道:``红娘你且先行,教小姐权时落后。''

[夫人云]他是个女孩儿家,着他落后怎么?

[红唱][秃厮儿]我则道神针法灸,谁承望燕侣莺俦。他两个经今月余则是一处宿,何须你一一问缘由?

[圣药王]他每不识忧,不识愁,一双心意两相投。夫人得好休,便好休,这其间何必苦追求?常言道``女大不中留''。

[夫人云]这端事,都是你个贱人!

[红云]非是张生、小姐、红娘之罪,乃夫人之过也。

[夫人云]这贱人倒指下我来,怎么是我之过?

[红云]信者,人之根本,``人而无信,不知其可也。大车无輗,小车无軏,其何以行之哉?''当日军围普救,夫人所许退军者,以女妻之。张生非慕小姐颜色,岂肯建区区退军之策?兵退身安,夫人悔却前言,岂得不为失信乎?既然不肯成就其事,只合酬之以金帛,令张生舍此而去。却不当留请张生于书院,使怨女旷夫,各相早晚窥视,所以夫人有此一端。目下老夫人若不息其事,一来辱没相国家谱,二来张生日后名重天下,施恩于人,忍令反受其辱哉!使至官司,夫人亦得治家不严之罪。官司若推其详,亦知老夫人背义而忘恩,岂得为贤哉?红娘不敢自专,乞望夫人台鉴:莫若恕其小过,成就大事,撋之以去其污,岂不为长便乎?

[麻郎儿]秀才是文章魁首,姐姐是仕女班头;一个通彻三教九流,一个晓尽描鸾刺绣。

[幺篇]世有、便休、罢手,大恩人怎做敌头?起白马将军故友,斩飞虎叛贼草寇。

[络丝娘]不争和张解元参辰卯酉,便是与崔相国出乖弄丑。到底干连着自己骨肉,夫人索穷究。

[夫人云]这小贱人也道得是。我不合养了这个不肖之女。待经官呵,玷辱家门。罢,罢,俺家无犯法之男,再婚之女,与了这厮罢!红娘,唤那贱人来!

[红见旦云]且喜姐姐,那棍子则是滴溜溜在我身上,吃我直说过了。我也怕不得许多。夫人如今唤你来,待成合亲事。

[旦云]羞人答答的,怎么见夫人?

[红云]娘根前有甚么羞!

[小桃红]当日个月明才上柳梢头,却早人约黄昏后。羞得我脑背后将牙儿衬着衫儿袖。猛凝眸,看时节则见鞋底尖儿瘦。一个恣情的不休,一个哑声儿厮耨。呸!那其间可怎生不害半星儿羞?

[旦见夫人科]

[夫人云]莺莺,我怎生抬举你来?今日做这等的勾当!则是我的孽障,待怨谁的是!我待经官来,辱没了你父亲,这等事,不是俺相国人家的勾当。罢罢罢,谁似俺养女的不长俊!红娘,书房里唤将那禽兽来!

[红唤末科]

[末云]小娘子,唤小生做甚么?

[红云]你的事发了也。如今夫人唤你来,将小姐配与你哩。小姐先招了也,你过去。

[末云]小生徨恐,如何见老夫人?当初谁在老夫人行说来?

[红云]休佯小心,过去便了。

[小桃红]既然泄漏怎干休,是我相投首。俺家里陪酒陪茶倒撋就,你休愁,何须约定通媒媾?我弃了部署不收,你原来``苗而不秀''。呸!你是个银样镴枪头。

[末见夫人科]

[夫人云]好秀才呵!岂不闻``非先王之德行不敢行''?我待送你去官司里去来,恐辱没了俺家谱。我如今将莺莺与你为妻,则是俺三辈儿不招白衣女婿,你明日便上朝取应去,我与你养着媳妇。得官呵,来见我;驳落呵,休来见我。

[红云]张生早则喜也。

[东原乐]相思事,一笔勾,早则展放从前眉儿皱,美爱幽欢恰动头。既能够,张生,你觑兀的般可喜娘庞儿也要人消受。

[夫人云]明日收拾行装,安排果酒,请长老一同送张生,到十里长亭去。

[旦念]寄语西河堤畔柳,安排青眼送行人。[同夫人下]

[红唱][收尾]来时节画堂箫鼓鸣春昼,列着一对儿鸾交凤友。那其间才受你说媒红,方吃你谢亲酒。

[并下]

\nchapter{第三折}

[夫人长老上云]今日送张生赴京,十里长亭安排下筵席。我和长老先行,不见张生、小姐来到。

[旦末红同上]

[旦云]今日送张生上朝取应,早是离人伤感,况值那暮秋天气,好烦恼人也呵!悲欢聚散一杯酒,南北东西万里程。

[正宫][端正好]碧云天,黄花地,西风紧,北雁南飞。晓来谁染霜林醉?总是离人泪。

[滚绣球]恨相见得迟,怨归去得疾。柳丝长玉骢难系。恨不倩疏林挂住斜晖。马儿迍迍的行,车儿快快的随,却告了相思回避,破题儿又早别离。听得道一声``去也'',松了金钏;遥望见十里长亭,减了玉肌。此恨谁知!

[红云]姐姐,今日怎么不打扮?

[旦云]你那知我的心里呵!

[叨叨令]见安排着车儿、马儿,不由人熬熬煎煎的气;有甚么心情花儿、厣儿,打扮得娇娇滴滴的媚;准备着被儿、枕儿,则索昏昏沉沉的睡;从今后衫儿、袖儿,都榅做重重叠叠的泪。兀的不闷杀人也么哥,兀的不闷杀人也么哥!久已后书儿、信儿,索与我恓恓惶惶的寄。[做到见夫人科]

[夫人云]张生和长老坐,小姐这壁坐,红娘将酒来。张生,你向前来,是自家亲眷,不要回避。俺今日将莺莺与你,到京师休辱末了俺孩儿,挣揣一个状元回来者。

[末云]小生托夫人余荫,凭着胸中之才,视官如拾芥耳。

[洁云]夫人主见不差,张生不是落后的人。

[把酒了,坐]

[旦长吁科][脱布衫]下西风黄叶纷飞,染寒烟衰草萋迷。酒席上斜签着坐的,蹙愁眉死临侵地。

[小梁州]我见他阁泪汪汪不敢垂,恐怕人知;猛然见了把头低,长吁气,推整素罗衣。

[幺篇]虽然久后成佳配,奈时间怎不悲啼。意似痴,心如醉,昨宵今日,清减了小腰围。

[夫人云]小姐把盏者。

[红递酒,旦把盏长吁科云]请吃酒。

[上小楼]合欢未已,离愁相继。想着俺前暮私情,昨夜成亲,今日别离。我谂知这几日相思滋味,却原来此别离情更增十倍。

[幺篇]年少呵轻远别,情薄呵易弃掷。全不想腿儿相挨,脸儿相偎,手儿相携。你与俺崔相国做女婿,妻荣夫贵,但得一个并头莲,煞强如状元及第。

[夫人云]红娘把盏者。

[红把酒科]

[旦唱][满庭芳]供食太急,须臾对面;顷刻别离。若不是酒席间子母每当回避,有心待与他举案齐眉。虽然是厮守得一时半刻,也合着俺夫妻每共桌而食。眼底空留意,寻思起就里,险化做望夫石。

[红云]姐姐不曾吃早饭,饮一口儿汤水。

[旦云]红娘,甚么汤水咽得下。

[快活三]将来的酒共食,尝着似土和泥;假若便是土和泥,也有些土气息,泥滋味。

[朝天子]暖溶溶玉醅,白泠泠似水。多半是相思泪。眼面前茶饭怕不待要吃,恨塞满愁肠胃。蜗角虚名,蝇头微利,拆鸳鸯在两下里。一个这壁,一个那壁,一递一声长吁气。

[夫人云]辆起车儿,俺先回去,小姐随后和红娘来。[下]

[末辞洁科]

[洁云]此一行别无话儿,贫僧准备买登科录看,做亲的茶饭,少不得贫僧的。先生在意,鞍马上保重者。从今经忏无心礼,专听春雷第一声。[下]

[旦唱][四边静]霎时间杯盘狼籍,车儿投东,马儿向西,两意徘徊,落日山横翠。知他今宵宿在那里?在梦也难寻觅。张生,此一行得官不得官,疾便回来。

[末云]小生这一去,白夺一个状元。正是:青霄有路终须到,金榜无名誓不归。

[旦云]君行别无所谓,口占一绝,为君送行:弃掷今何在,当时且自亲。还将旧来意,怜取眼前人。

[末云]小姐之意差矣,张珙更敢怜谁?谨赓一绝,以剖寸心:人生长远别,孰与最关亲?不遇知音者,谁怜长叹人?

[旦唱][耍孩儿]淋漓襟袖啼红泪,比司马青衫更湿。伯劳东去燕西飞,未登程先问归期。虽然眼底人千里,且尽生前酒一杯。未饮心先醉,眼中流血,心内成灰。

[五煞]到京师服水土,趁程途节饮食,顺时自保揣身体。荒村雨露宜眠早,野店风霜要起迟。鞍马秋风里,最难调护,最要扶持。

[四煞]这忧愁诉与谁?相思只自知,老天不管人憔悴。泪添九曲黄河溢,恨压三峰华岳低。到晚来闷把西楼倚,见了些夕阳古道,衰柳长堤。

[三煞]笑吟吟一处来,哭啼啼独自归。归家若到罗帏里,昨宵个绣衾香暖留春住,今夜个翠被生寒有梦知。留恋你别无意,见据鞍上马,阁不住泪眼愁眉。

[末云]有甚言语,嘱咐小生咱?

[旦唱][二煞]你休忧文齐福不齐,我则怕你停妻再娶妻。休要一春鱼雁无消息,我这里青鸾有信频须寄,你却休金榜无名誓不归。此一节君须记:若见了那异乡花草,再休似此处栖迟。

[末云]再谁似小姐,小生又生此念?

[旦唱][一煞]青山隔送行,疏林不做美,淡烟暮霭相遮蔽。夕阳古道无人语,禾黍秋风听马嘶。我为甚么懒上车儿内?来时甚急,去后何迟!

[红云]夫人去好一会,姐姐,咱家去。

[旦唱][收尾]四围山色中,一鞭残照里。遍人间烦恼填胸臆,量这些大小车儿如何载得起?

[旦红下]

[末云]仆童,赶早行一程儿,早寻个宿处。泪随流水急,愁逐野云飞。[下]

\nchapter{第四折}

[末引仆骑马上开]离了蒲东早三十里也,兀的前面是草桥,店里宿一宵,明日赶早行。这马百般儿不肯走。行色一鞭催去马,羁愁万斛引新诗。

[双调][新水令]望蒲东萧寺暮云遮,惨离情半林黄叶。马迟人意懒,风急雁行斜。离恨重叠,破题儿第一夜。想着昨日受用,谁知今日凄凉!

[步步娇]昨夜个翠被香浓熏兰麝,欹珊枕把身躯儿趄。脸儿厮揾者,仔细端详,可憎的别。铺云鬓玉梳斜,恰便似半吐初生月。早至也。店小二哥那里?

[小二哥上云]官人,俺这头房里下。

[末云]琴童,接了马者。点上灯,我诸般不要吃,则要睡些儿。

[仆云]小人也辛苦,待歇息也。[在床前打铺做睡科]

[末云]今夜甚睡得到我眼里来也!

[落梅风]旅馆欹单枕,秋蛩鸣四野,助人愁的是纸窗儿风裂。乍孤眠被儿薄又怯,冷清清几时温热!

[末睡科]

[旦上云]长亭畔别了张生,好生放心不下。老夫人和梅香都睡了,我私奔出城,赶上和他同去。

[乔木查]走荒郊旷野,把不住心娇怯,喘吁吁难将两气接。疾忙赶上者,打草惊蛇。

[搅筝琶]他把我心肠扯,因此不避路途赊。瞒过俺能拘管的夫人,稳住俺厮齐攒的侍妾。想着他临上马痛伤嗟,哭得我也似痴呆。不是我心邪,自别离已后,到西日初斜,愁得来陡峻,瘦得来唓嗻。则离得半个日头,却早又宽掩过翠裙三四褶。谁曾经这般磨灭。

[锦上花]有限姻缘,方才宁贴;无奈功名,使人离缺。害不了的愁怀,却才觉些;掉不下的思量,如今又也。清霜净碧波,白露下黄叶。下下高高,道路曲折;四野风来左右乱踅。我这里奔驰,他何处困歇?

[清江引]呆答孩店房儿里没话说,闷对如年夜。暮雨催寒蛩,晓风吹残月,今宵酒醒何处也?

[旦云]在这个店儿里,不免敲门。

[末云]谁敲门哩?是一个女人的声音,我且开门看咱。这早晚是谁?

[庆宣和]是人呵疾忙快分说,是鬼呵合速灭。

[旦云]是我。老夫人睡了,想你去了呵,几时再得见,特来和你同去。

[末唱]听说罢将香罗袖儿拽,却原来是姐姐、姐姐。难得小姐的心勤!

[乔牌儿]你是为人须为彻,将衣袂不藉。绣鞋儿被露水泥沾惹,脚心儿管踏破也。

[旦云]我为足下呵,顾不得迢递。

[旦唧唧了][甜水令]想着你废寝忘餐,香消玉减,花开花谢,犹自觉争些。便枕冷衾寒,凤只鸾孤,月圆云遮,寻思来有甚伤嗟?

[折桂令]想人生最苦离别!可怜见千里关山,独自跋涉。似这般割肚牵肠,倒不如义断恩绝。虽然是一时间花残月缺,休猜做瓶坠簪折。不恋豪杰,不羡骄奢,生则同衾,死则同穴。

[外净一行扮卒子上叫云]恰才见一女子渡河,不知那里去了,打起火把者!分明见他走在这店中去也。将出来!将出来!

[末云]却怎了?

[旦云]你近后,我自开门对他说。

[水仙子]硬围着普救寺下锹撅,强当住咽喉仗剑钺。贼心肠馋眼脑天生得劣。

[卒子云]你是谁家女子,夤夜渡河?

[旦唱]休言语,靠后些!杜将军你知道他是英杰,觑不觑着你为了醯酱,指一指教你化做膋血——骑着匹白马来也。

[卒子抢旦下]

[末惊觉云]呀,原来却是梦里。且将门儿推开看,只见一天露气,满地霜华,晓星初上,残月犹明。无端喜鹊高枝上,一枕鸳鸯梦不成。

[雁儿落]绿依依墙高柳半遮,静悄悄门掩清秋夜,疏剌剌林梢落叶风,昏惨惨云际穿窗月。

[得胜令]惊觉我的是颤巍巍竹影走龙蛇,虚飘飘庄周梦蝴蝶,絮叨叨促织儿无休歇,韵悠悠砧声儿不断绝。痛煞煞伤别,急剪剪好梦儿应难舍;冷清清的咨嗟,娇嘀嘀玉人儿何处也?

[仆云]天明也,咱早行一程儿,前面打火去。

[末云]店小二哥,还你房钱,鞴了马者。

[鸳鸯煞]柳丝长咫尺情牵惹,水声幽仿佛人呜咽。斜月残灯,半明不灭。唱道是旧恨连绵,新愁郁结;恨塞离愁,满肺腑难淘泻。除纸笔代喉舌,千种相思对谁说。

[并下]

[络丝娘煞尾]都则为一官半职,阻隔得千山万水。

题目:小红娘成好事\ 老夫人问私情

正名:短长亭斟别酒\ 草桥店梦莺莺

\npart{第五本\ 张君瑞庆团圆杂剧}

\nchapter{楔子}

[末引仆人上开云]自暮秋与小姐相别,倏经半载之际。托赖祖宗之荫,一举及第,得了头名状元。如今在客馆,听候圣旨御笔除授。惟恐小姐挂念,且修一封书,令琴童家去,达知夫人,便知小生得中,以安其心。琴童过来,你将文房四宝来,我写就家书一封,与我星夜到河中府去。见小姐时,说:``官人怕娘子忧,特地先着小人将书来。''即忙接了回书来者。过日月好疾也呵!

[仙吕][赏花时]相见时红雨纷纷点绿苔,别离后黄叶萧萧凝暮霭。今日见梅开,别离半载。琴童,我嘱咐你的言语记着:则说道特地寄书来。[下]

[仆云]得了这书,星夜望河中府走一遭。[下]

\nchapter{第一折}

[旦引红娘上开云]自张生去京师,不觉半年,杳无音信。这些时神思不快,妆镜懒抬,腰肢瘦损,茜裙宽褪,好烦恼人也呵!

[商调][集贤宾]虽离了我眼前,却在心上有;不甫能离了心上,又早眉头。忘了时依然还又,恶思量无了无休。大都来一寸眉峰,怎当他许多颦皱?新愁近来接着旧愁,厮混了难分新旧。旧愁似太行山隐隐,新愁似天堑水悠悠。

[红云]姐姐往常针尖不倒,其实不曾闲了一个绣床,如今百般的闷倦。往常也曾不快,将息便可,不似这一场,清减得十他利害。

[旦唱][逍遥乐]曾经消瘦,每遍犹闲,这番最陡。

[红云]姐姐心儿闷呵,那里散心耍咱。

[旦唱]何处忘忧?看时节独上妆楼,手卷珠帘上玉钩,空目断山明水秀。见苍烟迷时树,衰草连天,野渡横舟。

[旦云]红娘,我这衣裳,这些时都不似我穿的。

[红云]姐姐,正是``腰细不胜衣''。

[旦唱][挂金索]裙染榴花,睡损胭脂皱;纽结丁香,掩过芙蓉扣;线脱珍珠,泪湿香罗袖;杨柳眉颦,人比黄花瘦。

[仆人上云]奉相公言语,特将书来与小姐。恰才前厅上见了夫人,夫人好生欢喜,着入来见小姐,早至后堂。[咳嗽科]

[红问云]谁在外面?[见科][红见仆人]

[红笑云]你几时来?可知道昨夜灯花报,今朝喜鹊噪。姐姐正烦恼哩。你自来?和哥哥来?

[仆云]哥哥得了官也,着我寄书来。

[红云]你则在这里等着,我对俺姐姐说了呵,你进来。

[红见旦笑科]

[旦云]这小妮子怎么?

[红云]姐姐大喜,大喜!咱姐夫得了官也!

[旦云]这妮子见我闷呵,特故哄我。

[红云]琴童在门首,见了夫人了,使他进来见姐姐,姐夫有书。

[旦云]惭愧,我也有盼着他的日头!唤他入来。

[仆入见旦科]

[旦云]琴童,你几时离京师?

[仆云]离京一月多也。我来时,哥哥去吃游街棍子去了。

[旦云]这禽兽不省得,状元唤做夸官,游街三日。

[仆云]夫人说的便是。有书在此。

[旦做接书科][金菊花]早是我只因他去减了风流,不争你寄得书来又与我添些儿证候。说来的话儿不应口,无语低头,书在手,泪凝眸。

[旦开书看科][醋葫芦]我这里开时和泪开,他那里修时和泪修,多管阁着笔尖儿未写早泪先流,寄来的书泪点儿兀自有。我将这新痕把旧痕湮透,正是一重愁翻做两重愁。

[旦念书科]``张珙百拜,奉启芳卿可人妆次:自暮秋拜违,倏尔半载。上赖祖宗之荫,下托贤妻之德,举中甲第。即日于招贤馆寄迹,以伺圣旨御笔除授。惟恐夫人与贤妻忧念,特令琴童奉书驰报,庶几免虑。小生身虽遥而心常迩矣,恨不得鹣鹣比翼,邛邛并躯。重功名而薄恩爱者,诚有浅见贪饕之罪。他日面会,自当请谢不备。后成一绝,以奉清照:玉京仙府探花郎,寄语蒲东窈窕娘。指日拜恩衣昼锦,定须休作倚门妆。''

[幺篇]当日向西厢月底潜,今日向琼林宴上搊。谁承望跳东墙脚步占了鳌头?怎想道惜花心养成折桂手?脂粉丛里包藏着锦绣?从今后晚妆楼改做了至公楼!

[旦云]你吃饭不曾?

[仆云]上告夫人知道:早晨至今,空立厅前,那有饭吃?

[旦云]红娘,你快取饭与他吃。

[仆云]感蒙赏赐,我每就此吃饭。夫人写书,哥哥着小人索了夫人回书,至紧,至紧。

[旦云]红娘,将笔砚来。

[红将来科]

[旦云]书却写了,无可表意。只有汗衫一领,裹肚一条,袜儿一双,瑶琴一张,玉簪一枚,斑管一枝。琴童,你收拾得好者。红娘,取银十两来,就与他盘缠。

[红娘云]姐夫得了官,岂无这几件东西,寄与他有甚缘故?

[旦云]你不知道,这汗衫儿呵——

[梧叶儿]他若是和衣卧,便是和我一处宿;但贴着他皮肉,不信不想我温柔。

[红云]这裹肚要怎么?

[旦唱]常则不要离了前后,守着他左右,紧紧的系在心头。

[红云]这袜儿如何?

[旦唱]拘管他胡行乱走。

[红云]这琴他那里自有,又将去怎么?

[旦唱][后庭花]当日五言诗紧趁逐,后来因七弦琴成配偶。他怎肯冷落了诗中意,我则怕生疏了弦上手。

[红云]玉簪呵,有甚主意?

[旦唱]我须有个缘由,他如今功名成就,则怕他撇人在脑背后。

[红云]斑管,要怎的?

[旦唱]湘江两岸秋,当日娥皇因虞舜愁,今日莺莺为君瑞忧。这九嶷山下竹,共香罗衫袖口——

[青哥儿]都一般啼痕湮透。似这等泪斑宛然依旧,万古情缘一样愁。涕泪交流,怨慕难收。对学士叮咛说缘由,是必休忘旧。

[旦云]琴童,这东西收拾好者。

[仆云]理会得。

[旦唱][醋葫芦]你逐宵野店上宿,休将包袱做枕头,怕油脂腻展污了恐难酬。倘或水浸雨湿休便扭,我则怕干时节熨不开褶皱。一桩桩一件件细收留。

[金菊花]书封雁足此时修,情系人心早晚休?长安望来天际头,倚遍西楼,人不见,水空流。

[仆云]小人拜辞,即便去也。

[旦云]琴童,你见官人对他说。

[仆云]说甚么?

[旦唱][浪里来煞]他那里为我愁,我这里因他瘦。临行时啜赚人的巧舌头:指归期约定九月九,不觉的过了小春时候。到如今悔教夫婿觅封侯。

[仆云]得了回书,星夜回俺哥哥话去。[下]

\nchapter{第二折}

[末上云]画虎未成君莫笑,安排牙爪始惊人。本是举过便除,奉圣旨,着翰林院编修国史。他每那知我的心,甚么文章做得成!使琴童递佳音,不见回来。这几日睡卧不宁,饮食少进,给假在驿亭中将息。早间太医院着人来看视,下药去了。我这病,卢扁也医不得。自离了小姐,无一日心闲也呵!

[中吕][粉蝶儿]从到京师,思量心旦夕如是,向心头横躺着俺那莺儿。请医师,看诊罢,一星星说是。本意待推辞,则被他察虚实不须看视。

[醉春风]他道是医杂证有方术,治相思无药饵。莺莺,你若是知我害相思,我甘心儿死、死。四海无家,一身客寄,半年将至。

[仆上云]我则道哥哥除了,原来在驿亭中抱病。须索回书去咱。

[见了科]

[末云]你回来了也。

[迎仙客]疑怪这噪花枝灵鹊儿,垂帘幕喜蛛儿,正应着短檠上夜来灯爆时。若不是断肠词,决定是断肠诗。

[仆云]小夫人有书至此。

[末接科]写时管情泪如丝。既不呵,怎生泪点儿封皮上渍?

[末读书科]``薄命妾崔氏拜覆,敬奉才郎君瑞文几:自音容去后,不觉许时,仰敬之心,未尝少怠。纵云日近长安远,何故鳞鸿之杳矣?莫因花柳之心,弃妾恩情之意。正念间,琴童至,得见翰墨,始知中科,使妾喜之如狂。郎之才望,亦不辱相国之家谱也。今因琴童回,无以奉贡,聊有瑶琴一张,玉簪一枚,斑管一枚,裹肚一条,汗衫一领,袜儿一双,权表妾之真诚。匆匆草字欠恭,伏乞情恕不备。谨依来韵,遂继一绝云:阑干倚遍盼才郎,莫恋宸京黄四娘。病里得书如中甲,窗前览镜试新妆。''那风风流流的姐姐!似这等女子,张珙死也得着了。

[上小楼]这的堪为字史,当为款识。有柳骨颜筋,张旭张芝,羲之献之。此一时,彼一时,佳人才思,俺莺莺世间无二。

[幺篇]俺做经咒般持,符箓般使。高似金章,重似金帛,贵似金资。这上面若签个押字,使个令史,差个勾使,则是一张忙不及印赴期的咨示。

[末拿汗衫儿科]休说文章,则看他这针黹,人间少有。

[满庭芳]怎不教张生爱尔,堪针工出色,女教为师。几千般用意针针是,可索寻思。长共短又没个样子,窄和宽想象著腰肢,好共歹无人试。想当初做时,用煞那小心儿。小姐寄来这几件东西,都有缘故,一件件我都猜着。

[白鹤子]这琴,他教我闭门学禁指,留意谱声诗。调养圣贤心,洗荡巢由耳。

[二]这玉簪,纤长如竹笋,细白似葱枝,温润有清香,莹洁无瑕眦。

[三]这斑管,霜枝曾栖凤凰,泪点渍胭脂。当时舜帝恸娥皇,今日淑女思君子。

[四]这裹肚,手中一叶绵,灯下几回丝,表出腹中愁,果称心间事。

[五]这鞋袜儿,针脚儿细似虮子,绢帛儿腻似鹅脂,既知礼不胡行,愿足下当如此。琴童,你临行,小夫人对你说甚么?

[仆云]着哥哥休别继良姻。

[末云]小姐,你尚然不知我的心哩!

[快活三]冷清清客店儿,风淅淅雨丝丝,雨儿零风儿细梦回时,多少伤心事!

[朝天子]四肢不能动止,急切里盼不到蒲东寺。小夫人须是你见时,别有甚闲传示?我是个浪子官人,风流学士,怎肯带残花折旧枝。自从、到此,甚的是闲街市。

[贺圣朝]少甚宰相人家,招婿的娇姿?其间或有个人儿似尔,那里取那温柔,这般才思?想莺莺意儿,怎不教人梦想眠思。琴童来,将这衣裳东西收拾好者。

[耍孩儿]则在书房中倾倒个藤箱子,向箱子里面铺几张纸。放时节须索用心思,休教藤刺儿抓住绵丝。高抬在衣架上怕吹了颜色,乱穰在包袱中恐剉了褶儿。当如此,切须爱护,勿得因而。

[二煞]恰新婚才燕尔,为功名来到此。长安忆念蒲东寺。昨宵爱春风桃李花开夜,今日愁秋雨梧桐叶落时。愁如是,身遥心迩,坐想行思。

[三煞]这天高地厚情,直到海枯石烂时。此时作念何时止,直到烛灰眼下才无泪,蚕老心中罢却丝。我不比游荡轻薄子,轻夫妇的琴瑟,拆鸾凤的雄雌。

[四煞]不闻黄犬音,难传红叶诗,驿长不遇梅花使。孤身去国三千里,一日归必十二时。凭栏视,听江声浩荡,看山色参差。

[尾]忧则忧我在病中,喜则喜你来到此。投至得引人魂卓氏音书至,险将这害鬼病的相如盼望死。[下]

\nchapter{第三折}

[净扮郑恒上开云]自家姓郑,名恒,字伯常。先人拜礼部尚书,不幸早丧。后数年,又丧母。先人在时,曾定下俺姑娘的女孩儿莺莺为妻,不想姑夫亡化,莺莺孝服未满,不曾成亲。俺姑娘将着这灵榇,引着莺莺,回博陵下葬。为因路阻,不能得去。数月前写书来,唤我同扶柩去。因家中无人,来得迟了。我离京师,来到河中府,打听得孙飞虎欲掳莺莺为妻,得一个张君瑞退了贼兵。俺姑娘许了他。我如今到这里,没这个消息便好去见他;既有这个消息,我便撞将去呵,没意思。这一件事,都在红娘身上。我着人去唤他,则说``哥哥从京师来,不敢来见姑娘,着红娘来下处来,有话去对姑娘行说去。''去的人好一会了,不见来。见姑娘和他有话说。

[红上云]郑恒哥哥在下处,不来见夫人,却唤我说话。夫人着我来,看他说甚么。

[见净科]哥哥万福。夫人道:``哥哥来到呵,怎么不来家里来?''

[净云]我有甚颜色见姑娘?我唤你来的缘故是怎生?当日姑夫在时,曾许下这门亲事。我今番到这里,姑夫孝已满了,特地央及你去夫人行说知,拣一个吉日,了这件事,好和小姐一答里下葬去。不争不成合,一答里路上难厮见。若说得肯呵,我重重的相谢你。

[红云]这一节话再也休题。莺莺已与了别人了也。

[净云]道不得``一马不跨双鞍''!可怎生父在时曾许了我,父丧之后母倒悔亲?这个道理那里有?

[红云]却非如此说。当日孙飞虎将半万贼兵来时,哥哥你在那里?若不是那生呵,那里得俺一家儿来?今日太平无事,却来争亲;倘被贼人掳去呵,哥哥如何去争?

[净云]与了一个富家,也不枉了,却与了这个穷酸饿醋。偏我不如他?我仁者能仁、身里出身的的根脚,又是亲上做亲,况兼他父命。

[红云]他倒不如你?噤声!

[越调][斗鹌鹑]卖弄你仁者能仁,倚仗你身里出身;至如你官上加官,也不合亲上做亲。又不曾执羔雁邀媒,献币帛问肯。恰洗了尘,便待要过门。枉腌了他金屋银屏,枉污了他锦衾绣裀。

[紫花儿序]枉蠢了他梳云掠月,枉羞了他惜玉怜香,枉村了他殢雨尤云。当日三才始判,两仪初分;乾坤,清者为乾,浊者为坤,人在中间相混。君瑞是君子清贤,郑恒是小人浊民。

[净云]贼来,怎地他一个人退得?都是胡说!

[红云]我对你说。

[天净沙]看河桥飞虎将军,叛蒲东掳掠人民,半万贼屯合寺门,手横着霜刃,高叫道要莺莺做压寨夫人。

[净云]半万贼,他一个人济甚么事?

[红云]贼围之甚迫,夫人慌了,和长老商议,拍手高叫:``两廊不问僧俗,如退得贼兵的,便将莺莺与他为妻。''忽有游客张生,应声而前曰:``我有退兵之策,何不问我?''夫人大喜,就问其计何在。生云:``我有一故人白马将军,见统十万之众,镇守蒲关。我修书一封,着人寄去,必来救我。''不想书至兵来,其困即解。

[小桃红]洛阳才子善属文,火急修书信。白马将军到时分,灭了烟尘。夫人小姐都心顺,则为他威而不猛,言而有信,因此上不敢慢于人。

[净云]我自来未尝闻其名,知他会也不会!你这个小妮子,卖弄他偌多!

[红云]便又骂我!

[金蕉叶]他凭着讲性理《齐论》《鲁论》,作词赋韩文柳文,他识道理为人敬人,掩家里有信行知恩报恩。

[调笑令]你值一分,他值百十分,萤火焉能比月轮?高低远近都休论,我拆白道字辨与你个清浑。

[净云]这小妮子省得甚么拆白道字?你拆与我听。

[红唱]君端是个``肖''字这壁着个``立人'',你是个``木寸''``马户''``尸巾''。

[净云]木寸、马户、尸巾,你道我是个``村驴𡱃''?我祖代是相国之门,到不如你个白衣饿夫穷士?做官的则是做官!

[红唱][秃厮儿]他凭师友君子务本,你倚父兄仗势欺人。齑盐日月不嫌贫,治百姓新民、传闻。

[圣药王]这厮乔议论,有向顺。你道是官人则合做官人,信口喷,不本分。你道穷民到老是穷民,却不道``将相出寒门''!

[净云]这桩事,都是那长老秃驴弟子孩儿,我明日慢慢的和他说话。

[红唱][麻儿郎]他出家儿慈悲为本,方便为门。横死眼不识好人,招祸口不知分寸。

[净云]这是姑夫的遗留,我拣日,牵羊担酒,上门去,看姑娘怎么发落我!

[红唱][幺篇]讪筋,发村,使狠,甚的是软款温存。硬打捱强为眷姻,不睹事强谐秦晋。

[净云]姑娘若不肯,着二三十个伴当,抬上轿子,到下处脱了衣裳,赶将来,还你一个婆娘!

[红唱][络丝娘]你须是郑相国嫡亲的舍人,须不是孙飞虎家生的莽军。乔嘴脸、腌躯老、死身分,少不得有家难奔。

[净云]兀的那小妮子,眼见得受了招安了也。我也不对你说,明日我要娶,我要娶!

[红云]不嫁你,不嫁你!

[收尾]佳人有意郎君俊,我待不喝采其实怎忍。

[净云]你喝一声我听。

[红笑云]你这般颓嘴脸,则好偷韩寿下风头香,傅何郎左壁厢粉。[下]

[净脱衣科云]这妮子拟定都和那酸丁演撒!我明日自上门去见俺姑娘,则做不知。我则道:``张生赘在卫尚书家,做了女婿。''俺姑娘最听是非,他自小又爱我,必有话说。休说别个,则这一套衣服也冲动他。自小京师同住,惯会寻章摘句。姑夫许我成亲,谁敢将言相拒?我若放起刁来,且看莺莺那去!且将压善欺良意,权作尤云殢雨心。[下]

[夫人上云]夜来郑恒至,不来见我,唤红娘去问亲事。据我的心,则是与孩儿是;况兼相国在时已许下了。我便是违了先夫的言语。做我一个主家的不着,这厮每做下来。拟定则与郑恒,他有言语,怪他不得也。料持下酒者,今日他敢来见我也。

[净上云]来到也,不索报覆,自入去见夫人。[拜夫人哭科]

[夫人云]孩儿,既来到这里,怎么不来见我?

[净云]小孩儿有甚嘴脸来见姑娘!

[夫人云]莺莺为孙飞虎一节,等你不来,无可解危,许张生也。

[净云]那个张生?敢便是状元?我在京师看榜来,年纪有二十四五岁,洛阳张珙,夸官游街三日。第二日,头答正来到卫尚书家门首,尚书的小姐十八岁也,结着彩楼,在那御街上,则一球正打着他。我也骑着马看,险些打着我。他家粗使梅香十余人,把那张生横拖倒拽入去。他口叫道:``我自有妻,我是崔相国家女婿!''那尚书有权势气象,那里听?则管拖将入去了。这个却才便是他本分,出于无奈。尚书说道:``我女奉圣旨,结彩楼,你着崔小姐做次妻。他是先奸后娶的,不应娶他。''闹动京师,因此认得他。

[夫人怒云]我道这秀才不中抬举,今日果然负了俺家。俺相国之家,世无与人做次妻之理。既然张生奉圣旨娶了妻,孩儿,你拣个吉日良辰,依着姑夫的言语,依旧入来做女婿者。

[净云]倘或张生有言语,怎生?

[夫人云]放着我哩。明日拣个吉日良辰,你便过门来。[下]

[净云]中了我的计策了。准备筵席茶礼花红,克日过门者。[下]

[洁上云]老僧昨日买登科记看来,张生头名状元,授着河中府尹。谁想夫人没主张,又许了郑恒亲事。老夫人不肯去接,我将着肴馔,直至十里长亭,接官走一遭。[下]

[杜将军上云]奉圣旨,着小官主兵蒲关,提调河中府事,上马管军,下马管民。谁想君瑞兄弟一举及第,正授河中府尹,不曾接得。眼见得在老夫人宅里下,拟定乘此机会成亲。小官牵羊担洒,直至老夫人宅上,一来庆贺状元,二来做主亲,与兄弟成此大事。左右那里?将马来,到河中府走一遭。[下]

\nchapter{第四折}

[夫人上云]谁想张生负了俺家,去卫尚书家做女婿去。今日不负老相公遗言,还招郑恒为婿。今日好个日子,过门者。准备下筵席,郑恒敢待来也。

[末上云]小官奉圣旨,正授河中府尹。今日衣锦还乡,小姐的金冠霞帔都将著,若见呵,双手索送过去。谁想有今日也呵!文章旧冠乾坤内,姓字新闻日月边。

[双调][新水令]玉鞭骄马出皇都,畅风流玉堂人物。今朝三品职,昨日一寒儒。御笔亲除,将名姓翰林注。

[驻马听]张珙如愚,酬志了三尺龙泉万卷书;莺莺有福,稳请了五花官诰七香车。身荣难忘借僧居,愁来犹记题诗处。从应举,梦魂儿不离了蒲东路。

[末云]接了马者。

[见夫人科]新状元河中府尹婿张珙参见。

[夫人云]休拜,休拜!你是奉圣旨的女婿,我怎消受得你拜!

[末唱][乔牌儿]我谨躬身问起居,夫人这慈色为谁怒?我则见丫鬟使数都厮觑,莫不我身边有甚事故?

[末云]小生去时,夫人亲自饯行,喜不自胜。今日中选得官,夫人反行不悦,何也?

[夫人云]你如今那里想着俺家?道不得个``靡不有初,鲜克有终''。我一个女孩儿,虽然妆残貌陋,他父为前朝相国,若非贼来,足下甚气力到得俺家?今日一旦置之度外,却于卫尚书家作婿,岂有是理!

[末云]夫人听谁说?若有此事,天不盖,地不载,害老大小疔疮!

[雁儿落]若说着丝鞭仕女图,端的是塞满章台路。小生呵此间怀旧恩,怎肯别处寻亲去。

[得胜令]岂不闻``君子断其初'',我怎肯忘得有恩处?那一个贼畜生行嫉妒,走将来老夫人行厮间阻?不能勾娇姝,早共晚施心数;说来的无徒,迟和疾上木驴。

[夫人云]是郑恒说来,绣球儿打着马了,做女婿也。你不信呵,唤红娘来问。

[红上云]我巴不得见他。原来得官回来,惭愧,这是非对着也。

[末背问云]红娘,小姐好么?

[红云]为你别做了女婿,俺小姐依旧嫁了郑恒也。

[末云]有这般跷蹊的事!

[庆东原]那里有粪堆上长出连枝树,淤泥中生出比目鱼,不明白展污了姻缘簿?莺莺呵,你嫁个油炸猢猻的丈夫;红娘呵,你伏侍个烟薰猫儿的姐夫;张生呵,你撞着个水浸老鼠的姨夫。这厮坏了风俗,伤了时务。

[红唱][乔木查]妾前来拜覆,省可里心头怒。间别时来安乐否?你那新夫人何处居?比俺姐姐是何如?

[末云]和你也葫芦题了也。小生为小姐受过的苦,诸人不知,瞒不得你。不甫能成亲,焉有是理?

[搅争琶]小生若求了媳妇,则目下便身殂。怎肯忘得待月回廊,难撇下吹箫伴侣。受了些活地狱,下了些死工夫。不甫能得做妻夫,现将着夫人诰敕,县君名称,怎生待欢天喜地,两只手儿分付与,你刬地倒把人赃诬。

[红对夫人云]我道张生不是这般人,则唤小姐出来自问他。

[叫旦科]姐姐,快来问张生,我不信他直恁般薄情。叫见他呵,怒气冲天,实有缘故。

[旦见末科]

[末云]小姐间别无恙?

[旦云]先生万福。

[红云]姐姐有的言语,和他说破。

[旦长吁云]待说甚么的是!

[沉醉东风]不见时准备着千言万语,得相逢都变做短叹长吁。他急攘攘却才来,我羞答答怎生觑。将腹中愁恰待申诉,及至相逢一句也无。只道个``先生万福''。

[旦云]张生,俺家何负足下?足下见弃妾身,去卫尚书家为婿,此理安在?

[末云]谁说来?

[旦云]郑恒在夫人行说来。

[末云]小姐如何听这厮?张珙之心,惟天可表!

[落梅风]从离了蒲东路,来到京兆府,见个佳人世不曾回顾。硬揣个卫尚书家女孩儿为了眷属,曾见他影儿的也教灭门绝户!

[末云]这一桩事都在红娘身上,我则将言语傍着他,看他说甚么。红娘,我问人来,说道你与小姐将简帖儿去唤郑恒来。

[红云]痴人!我不合与你作成,你便看得我一般了。

[甜水令]君瑞先生,不索踌躇,何须忧虑。那厮本意糊涂;俺家世清白,祖宗贤良,相国名誉。我怎肯他跟前寄简传书?

[折桂令]那吃敲才怕不口里嚼蛆,那厮待数黑论黄,恶紫夺朱。俺姐姐更做道软弱囊揣,怎嫁那不值钱人样豭驹。你个东君索与莺莺做主,怎肯将嫩枝柯折与樵夫。那厮本意嚣虚,将足下亏图,有口难言,气夯破胸脯。

[红云]张生,你若端的不曾做女婿呵,我去夫人跟前一力保你。等那厮来,你和他两个对证。

[红见夫人云]张生并不曾人家做女婿,都是郑恒谎,等他两个对证。

[夫人云]既然他不曾呵,等郑恒那厮来对证了呵,再做说话。

[洁上云]谁想张生不举成名,得了河中府尹。老僧一径到夫人那里庆贺。这门亲事,几时成就?当初也有老僧来,老夫人没主张,便待要与郑恒。若与了他,今日张生来,却怎生?

[洁见末叙寒温科][对夫人云]夫人今日却知老僧说的是,张生决不是那一等没行止的秀才。他如何敢忘了夫人?况兼杜将军是证见,如何悔得他这亲事?

[旦云]张生此一事,必得杜将军来方可。

[雁儿落]他曾笑孙庞真下愚,若是论贾马非英物,正授着征西元帅府,兼领着陕右河中路。

[得胜令]是咱前者护身符,今日有权术。来时节定把先生助,决将贼子诛。他不识亲疏,啜赚良人妇。你不辨贤愚,无毒不丈夫。

[夫人云]着小姐去卧房里去者。

[旦下]

[杜将军上云]下官离了蒲关,到普救寺,第一来庆贺兄弟咱;第二来就与兄弟成就了这亲事。

[末对将军云]小弟托兄长虎威,得中一举。今者回来,本待做亲。有夫人的侄儿郑恒,来夫人行说道,你兄弟在卫尚书家作赘了。夫人怒欲悔亲,依旧要将莺莺与郑恒,焉有此理?道不得个``烈女不更二夫''。

[将军云]此事夫人差矣。君瑞也是礼部尚书之子,况兼又得一举。夫人世不招白衣秀士,今日反欲罢亲,莫非理上不顺?

[夫人云]当初夫主在时,曾许下这厮,不想遇此一难,亏张生请将军来,杀退贼众。老身不负前言,欲招他为婿。不想郑恒说道,他在卫尚书家做了女婿也,因此上我怒他,依旧许了郑恒。

[将军云]他是贼心,可知道诽谤他。老夫人如何便信得他?

[净上云]打扮得整整齐齐的,则等做女婿。今日好日头,牵羊担酒,过门走一遭。

[末云]郑恒,你来怎么?

[净云]苦也!闻知状元回,特来贺喜。

[将军云]你这厮,怎么要诳骗良人的妻子,行不仁之事,我跟前有甚么话说?我闻奏朝廷,诛此贼子。

[末唱][落梅风]你硬撞入桃源路,不言个谁是主,被东君把你个蜜蜂拦住。不信呵去那绿杨影里听杜宇,一声声道``不如归去''。

[将军云]那厮若不去呵,祗候拿下。

[净云]不必拿,小人自退亲事与张生罢。

[夫人云]相公息怒,赶出去便罢。

[净云]罢,罢!要这性命怎么,不如触树身死。妻子空争不到头,风流自古恋风流。三寸气在千般用,一日无常万事休。[净倒科]

[夫人云]俺不曾逼死他,我是他亲姑娘,他又无父母,我做主葬了者。着唤莺莺出来,今日做个庆喜的茶饭,着他两口儿成合者。

[旦红上,末旦拜科]

[末唱][沽美酒]门迎着驷马车,户列着八椒图,四德三从宰相女,平生愿足,托赖着众亲故。

[太平令]若不是大恩人拨刀相助,怎能够好夫妻似水如鱼。得意也当时题柱,正酬了今生夫妇。自古、相女、配夫,新状元花生满路。

[使臣上科]

[末唱][锦上花]四海无虞,皆称臣庶;诸国来朝,万岁山呼;行迈羲轩,德过舜禹;圣策神机,仁文义武。朝中宰相贤,天下庶民富;万里河清,五谷成熟;户户安居,处处乐土;凤凰来仪,麒麟屡出。

[清江引]谢当今盛明唐圣主,敕赐为夫妇。永老无别离,万古常完聚,愿普天下有情的都成了眷属。

[随尾]则因月底联诗句,成就了怨女旷夫。显得有志的状元能,无情的郑恒苦。[下]

\newpage

题目:小琴童传捷报\ 崔莺莺寄汗衫

正名:郑伯常干舍命\ 张君瑞庆团圆

\chapter*{总目}

张君瑞要做东床婿\ 法本师住持南赡地\\
老夫人开宴北堂春\ 崔莺莺待月西厢记

\end{document}